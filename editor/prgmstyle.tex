%Copied from numeric.cbx to imitate numerical citations.
\providebool{bbx:subentry}
\newbibmacro*{citenum}{%Note: the original macro was called "cite". I did not redefine "cite", but instead defined a new macro "citenum", because the author-year citations use the "cite" macro too. "\renewbibmacro*{cite}" would have caused all the author-year citations to become numeric too.
  \printtext[bibhyperref]{%If you ever want to use hyperref
    \printfield{prefixnumber}%
    \printfield{labelnumber}%
    \ifbool{bbx:subentry}
      {\printfield{entrysetcount}}
      {}}}

\newbibmacro*{citetitle}{%
  \iffieldundef{shorthand}
  {\usebibmacro{cite:title}}%
  {\usebibmacro{cite:shorthand}}}

\newbibmacro*{cite:title}{%
  \printtext[bibhyperref]{%
    \printfield[citetitle]{labeltitle}}}

%Copied from numeric.cbx to define a new numeric citation command for @software entries.
\DeclareCiteCommand{\citeprgm}[\mkbibbrackets]
  {\usebibmacro{prenote}}
  {\usebibmacro{citeindex}%
   \usebibmacro{cite:title}}%
  {\multicitedelim}
  {\usebibmacro{postnote}}

% \begin{document}
% Cite books, \parencite{Book} \parencite{Book2}

% Cite online stuff, \citeprgm{Online} \citeprgm{Online2}

% %Print the normal author-year bibliography
% \printbibliography[nottype=online]

% %Redefine the bibliography environment to imitate the numeric citation style
% \defbibenvironment{bibliography}
% {\list
% {\printfield[labelnumberwidth]{labelnumber}}
% {\setlength{\labelwidth}{\labelnumberwidth}%
% \setlength{\leftmargin}{\labelwidth}%
% \setlength{\labelsep}{\biblabelsep}%
% \addtolength{\labelsep}{1em}
% \addtolength{\leftmargin}{\labelsep}%
% \setlength{\itemsep}{\bibitemsep}%
% \setlength{\parsep}{\bibparsep}}%
% \renewcommand*{\makelabel}[1]{\hss##1}}
% {\endlist}
% {\item}
% \DeclareFieldFormat{labelnumberwidth}{\mkbibbrackets{#1}\hspace{-1.1em}}
% \printbibliography[type=online,title=Online,resetnumbers]
% \end{document}