
\begin{partbacktext}
\part{Performance Assessment}\label{part:Performance}

The built environment is a collection of various types of assets that affect the well-being and quality of life of residents in an urban area. The list of such assets includes residential and commercial buildings, bridges, networks of roads, railways, pipelines and power lines, and their supporting facilities. The performance of these assets is quantified by Decision Variables (DV) that describe the influence of asset damage to the life of the affected community. As their name implies, these DVs are ultimately meant to drive decision- and policy-making.

The performance of assets heavily depends on the determination of the hazard and the asset response to a characteristic event that is consistent with the hazard at the asset location. Although this chapter focuses on performance assessment, some of the tools listed here have hazard and response estimation capabilities as well. Those features have been covered in the previous chapters and will not be mentioned here again. This chapter is organized around the types of assets or asset-networks needed to arrive at a description of the performance of an urban region. 

Seismic performance assessment of buildings has received a lot of attention from the research community and funding agencies in the past few decades \citep{ATC 1985; FEMA Mitigation Division 2018b; 2018c; fajfar2004performancebased, FEMA  2012}. Consequently, the most sophisticated and mature methods are available in that area (FEMA, 2012). Several researchers have focused on adopting these methods for other asset types \citep{werner2006redars, chmielewski2016response} and for other types of hazards \citep{FEMA Mitigation Division, 2018c, attary2017performancebased, barbato2013performancebased, lange2014application}. This is not a trivial task because damage and subsequent consequences can be fundamentally different for non-building assets and for non-seismic disasters.  

\end{partbacktext}
