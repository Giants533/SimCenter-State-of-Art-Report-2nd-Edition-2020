
\begin{partbacktext}
\part{Response Estimation}
\label{part:response}

Response estimation entails computational finite-element and other analysis methods to simulate the physical response of solids and fluids related to natural hazards engineering. The section on structural systems describes simulation technologies to analyze the response of constructed facilities (buildings, bridges, and other facilities) to the loading effects of gravity, earthquakes, storms (wind and storm-surge flows), and tsunami inundation. The section on geotechnical systems describes methods to explicitly simulate the detailed response of soil and soil-structure interaction under input ground motions. The simulation results are used to determine ground deformations, liquefaction, soil-structure interaction, and ground instabilities due to other phenomena (e.g., changes in ground water levels, scour, etc.). The sections on computational fluid dynamics address methods to simulate wind and water flows due to water inundation and tsunami. 

\end{partbacktext}