%%%%%%%%%%%%%%%%%%%% author.tex %%%%%%%%%%%%%%%%%%%%%%%%%%%%%%%%%%%
%
% sample root file for your "contribution" to a contributed volume
%
% Use this file as a template for your own input.
%
%%%%%%%%%%%%%%%% Springer %%%%%%%%%%%%%%%%%%%%%%%%%%%%%%%%%%%%%%%%%


%% RECOMMENDED %%%%%%%%%%%%%%%%%%%%%%%%%%%%%%%%%%%%%%%%%%%%%%%%%%%
%\documentclass[graybox]{svmult}
%
%% choose options for [] as required from the list
%% in the Reference Guide
%
%\usepackage{mathptmx}       % selects Times Roman as basic font
%\usepackage{helvet}         % selects Helvetica as sans-serif font
%\usepackage{courier}        % selects Courier as typewriter font
%\usepackage{type1cm}        % activate if the above 3 fonts are
                             % not available on your system
%
%\usepackage{makeidx}         % allows index generation
%\usepackage{graphicx}        % standard LaTeX graphics tool
%                             % when including figure files
%\usepackage{multicol}        % used for the two-column index
%\usepackage[bottom]{footmisc}% places footnotes at page bottom
%
%% see the list of further useful packages
%% in the Reference Guide
%
%\makeindex             % used for the subject index
%                       % please use the style svind.ist with
%                       % your makeindex program
%
%%%%%%%%%%%%%%%%%%%%%%%%%%%%%%%%%%%%%%%%%%%%%%%%%%%%%%%%%%%%%%%%%%%%%%%%%%%%%%%%%%%%%%%%%%
%
%\begin{document}

\title{Tropical Cyclone - Wind}
% Use \titlerunning{Short Title} for an abbreviated version of
% your contribution title if the original one is too long
\author{
    \textbf{Ahsan Kareem} 
    \and Liang Hu}
\tocauthor{}
\authorrunning{Kareem and Hu}
% Use \authorrunning{Short Title} for an abbreviated version of
% your contribution title if the original one is too long
%\institute{Name of First Author \at Name, Address of Institute, %\email{name@email.address}
%\and Name of Second Author \at Name, Address of Institute %\email{name@email.address}}
%
% Use the package "url.sty" to avoid
% problems with special characters
% used in your e-mail or web address
%
\maketitle

Extreme wind induced by tropical cyclones (TC - hurricane/typhoon/tropical storm) dominates the wind loading on structures in the U.S. coastal areas. To assess the damage, loss, and performance of buildings probabilistically under wind hazard as well as its secondary hazards (flood, rain, debris, storm surge, etc.), this section describes computational models and inputs available for estimating statistical characteristics of TC-induced wind speeds. 

Usually, field measurements of TC are limited and insufficient for estimating the probabilistic description of wind speeds; thus they are usually generated by Monte Carlo-based procedures. Such a simulation procedure starts from sampling input physical properties of a hurricane (e.g., intensity, track, and Holland B parameter) in terms of their individual probabilistic characteristics to simulate the wind field by which the wind speeds at a specific site may be recorded and estimated \citep{russell1969probability}. The simulation is carried out by employing phenomenological models of the hurricane wind field with random parameters. Other models based on meteorological aspects [e.g., MM5 \citep{liu1997multiscale} and WRF \citep{davis2008prediction}] are beyond the scope of this section. Three types of TC wind-field models are currently available. The first two models aim to solve the governing equation of motion of the TC atmospheric system directly using the central difference method: (I) height-resolved models, which are able to resolve the vertical structure of a tropical cyclone; and (II) slab models, which include an average or integration over the height of the governing equations. In contrast, the type III physics-based models solve the intensity and radial profile equations instead. The input variables are dependent upon the type of model selected.

\section{Input and Output Data}
\label{sec:storm_wind_io}

\paragraph{Measurements} Through the past two centuries, data from recorded TCs have been used by the wind engineering community to create and calibrate probabilistic models. The National Oceanic \& Atmospheric Administration (NOAA) provides an extended and comprehensive TC database for the Atlantic and Northeast Pacific, which encompasses data from reconnaissance and microwave and dropsonde radar, as well as anemometer measurements (NOAA Reanalysis Data 2018). Additional field measurements supplement this database [e.g., (Li et al., 2015b, wang2016measurements)].
\newline

\paragraph{Occurrence rate} This parameter describes the number of hurricanes that occur at a specific site. It is usually described by a Poisson or binomial distribution, whose parameters are obtained by statistics over the hurricane database \citep{li2016typhoon, vickery2009hurricane-c}.
\newline

\noindent\textbf{Track model: Initial location, Translation speed, and Heading} \\The track model describes the genesis point, heading direction, and translational speed of the center of a TC for simulation purposes. For a specific hurricane in the NOAA database, its best empirical track of the hurricane has already been synthesized by data fitting over various measurements. The database also describes how other TC parameters change along the track. For a specific site, the sub-region track model can be used, which only concerns the segment of TC tracks within a circle (often the radius is 500 km) centered at the site. This model is characterized by the perpendicular distance to the center and direction angle of the straight line track \citep{georgiou1986design, xiao2011typhoon}. However, the full track model is more popular in describing the genesis to dissipation of a TC because it enables the simulation of extreme TC winds simultaneously for a large region rather than a specific site. The genesis location can be randomly selected from the historical record or generated on the basis of its distribution function \citep{vickery2009hurricane-c}. Starting from the genesis location, the track is generated by Markov-type models, represented by auto-regressive functions in terms of TC parameters (latitude translation speed, sea surface temperature, etc.) as well as a random error term \citep{vickery2000simulation}, or by the Markovian transition probability function \citep{emanuel2006statistical}. The parameters of track models must be estimated from the hurricane database as well as other measurements (e.g., HadISST) \citep{li2016typhoon, liu2014projections, vickery2000simulation}. Recent investigations usually apply the kernel method for modeling those parameters \citep{chen2018statistical, mudd2015advancements}. Moreover, a dynamic track model (Beta-advection) has been developed based on isobaric wind speed measurements \citep{emanuel2006statistical}. 
\newline

\noindent\textbf{Intensity: Central pressure difference or maximum wind speed} \\Type I and II hurricane models use the central pressure difference as the proxy for the TC intensity. Here, an auto-regressive model for the TC relative intensity (a function in terms of the pressure difference) has been established along with the track models. The Type III model employs the maximum mean wind speed as the intensity measure, which may be predicted by a simple coupled ocean-atmosphere physical model CHIPS (Coupled Hurricane Intensity Prediction System) with its fast simulation algorithm \citep{emanuel2011selfstratification,emanuel2017fast,emanuel2004environmental} or by the historical record-free generator \citep{emanuel2008hurricanes}. 
\newline

\noindent\textbf{Size: Radius to maximum winds (RMW)} \\The Radius to Maximum Winds (RMW) denotes the size of a TC and is the only TC size parameter considered in Type I and II models. Type III models need additional parameters, e.g., the radius at the wind speed of $15.5$ m/s \citep{chavas2016model}. The probabilistic distribution of these size parameters can be estimated from the TC database. However, an empirical model of RMW has been developed in terms of the location and intensity parameters as well as a random error term \citep{vickery2008statistical, vickery2009hurricane-b}.
\newline

\noindent\textbf{Shape of radial profile: Holland B parameter} \\The B parameter was introduced by \cite{holland1980analytic} revised the radial pressure profile in Type I and II models to improve the goodness-of-fit of the maximum wind speed. From the TC database, this parameter can be estimated with respect to the reconnaissance data, which evolves with time. Similar to RMW, statistical models are available for B as a function of RMW and latitude \citep{powell2005state} or of a dimensionless function involving SST additionally \citep{vickery2008statistical}. A physics-based model has also been proposed by \cite{holland2008revised} and \cite{holland2010revised}.
\newline

\paragraph{Local terrain} The local topography at a specified site accounts for the boundary layer wind speed profile as well as the gust factor. A typical parameter of local terrain is the roughness length (or equivalently the shear velocity), which reflects the effects of upstream terrain within ~3 km on the near-ground winds. Calculating this parameter is challenging, especially in consideration of the rapid change of wind azimuth during a TC \citep{vickery2009hurricane-a}. It can be adopted from existing design codes/specifications and augmented by additional computations by taking average over various terrains along each wind direction. As long as field measurements of gust wind speeds are available at the specified site, the roughness length may also be estimated from the record \citep{masters2010objective}. Furthermore, the CFD-based method is also available to estimate the local wind characteristics with detailed modeling of surrounding terrains, which, though computationally inefficient, is expected to yield more accurate results and is often the only reliable method for complex terrain \citep{huang2013prediction,ishihara2005wind}. 
\newline

\paragraph{Landfall model parameters} After a hurricane makes landfall, the filling model starts to describe the weakening of the TC intensity, or, in other words, the increase in the central pressure difference. This model is typically an exponential decay function, whose decay constant is the filling rate as a statistical function in terms of the intensity, translational speed, and RMW \citep{vickery1995windfield, vickery2005simple, vickery2009hurricane-b}. Moreover, both the mean wind speed vertical profile and radial profile are subject to notable changes after landfall, which may be captured by recently developed empirical models (Fang et al., 2018b, snaiki2018semiempirical, zhao2013radial). 
\newline

\noindent\textbf{Output: Wind field} \\The main output of a TC simulation from an engineering perspective is the probabilistic model (CDF) of mean wind speed in any specified target location/region. A single hurricane scenario results in a mean wind speed and direction time history, usually at the 6-hour time interval. Additional effects of atmospheric turbulence may be reflected by gust factors as well as spectra. These results serve as the IM for the ensuing performance-based wind engineering analysis \citep{barbato2013performancebased, chuang2019efficient, liu2014projections, spence2014performancebased, unnikrishnan2016performancebased, xiao2011typhoon, yau2011hurricane}.

\section{Modeling Approaches}
\label{sec:storm_wind_methods}

Provided the inputs stated above, all TC wind field models aim at solving the steady mean wind speed from the 3D governing equation system describing atmospheric motion in a TC. Type I models solve the 3D motion equation system without any dimensional reduction; Type II and III models are, per se, 2D methods. Type II considers the equation system reduced from the original one, whereas the Type III model solves angular momentary equations derived by the physics-based mechanism of the TC rather than the original motion equation. Here, Type I models are able to solve wind speeds throughout the TC boundary layer height, whereas the other two models solve wind speeds at the gradient height, which are then converted to near-ground heights by the boundary layer wind speed profile. All the solved TC wind speeds need to be combined with the surface background wind speed, implying the use of the TC translational speed for Type I and II models. Eventually, the gust factor is applied to the mean wind speed to account for turbulence. Nonstationary effects associated with TC winds may also be considered.
\newline

\paragraph{Type I model} The basic atmospheric motion governing equation of TC is nonlinear and 3D, which can be solved numerically by the two-time-level time-split-based central difference scheme \citep{kepert2001dynamics-ii,kepert2011choosing}. It accounts for the salient height-related effects of both potential temperature and eddy viscosity (turbulent diffusivity represented by the vertical turbulent exchange coefficient K for momentum and heat) \citep{kepert2001dynamics-ii, kepert2010slab-ii}. Linearization of the nonlinear equation has been carried out considering the gradient balance wind speed to yield the surface horizontal momentum equations \citep{kepert2001dynamics-i}. The linearized equations are then solved by utilizing the perturbation method \citep{meng1995analytical} or the Fourier series expansion \citep{kepert2001dynamics-i}. Depending on the form of eddy viscosity (constant, height-dependent, or piece-wise linear or nonlinear) and the terms being neglected, various semi-analytical solutions are obtained (Fang et al., 2018a, huang2013prediction, kepert2006observed, meng1995analytical, meng1997numerical, snaiki2017linear). In comparison with the nonlinear solution, these linear solutions sacrifice accuracy to reduce computational costs \citep{kepert2014reply}. 
\newline

\paragraph{Type II model} By integrating the 3D equation over the vertical coordinate, a slab (or depth-averaged) model is derived \citep{kepert2010slab-i}. This model still involves the vertical turbulent diffusivity and is capable of calculating the vertical wind speed \citep{langousis2008extreme, smith1968surface, smith2008simple}. Further simplification is achieved by removing both the advective and/or diffusive fluxes at the upper boundary, leading to the common category of TC models popular in the structural engineering community \citep{powell2005state, shapiro1983asymmetric, vickery2000hurricane, vickery2009hurricane-b}. \cite{chow1971study} was the first to develop a central difference scheme to solve the model. Since then, other issues in this model have been addressed to enhance its applicability, e.g., the boundary layer, drag coefficient, track model, and approximate fast algorithm. So far, the parameters of this model have been well-recognized probabilistically based on the TC database \citep{vickery2008statistical}. A review paper guiding application of this model is also available \citep{vickery2009hurricane-a}. Note: the TC intensity and track inside this model are being updated \citep{mudd2015development, vickery2010synthetic}. 
\newline

\paragraph{Type III model} The foundation of this approach is a physics-based intensity model derived by regarding the TC as a Carnot heat engine \citep{emanuel2004tropical,emanuel1988maximum}. The maximum wind speed-represented intensity can be calculated along the track \citep{emanuel2011selfstratification}. Although a simple formula was used as the radial profile at the gradient height \citep{emanuel2006statistical,lin2012hurricane}, a more reliable model has been proposed by dividing the profile into its inner and outer regions. Physics-based expressions for the two regions have been derived and then joined by a differential equation system to establish the whole profile \citep{emanuel2004tropical,emanuel2011selfstratificationa}. Given the input, only the equation system of the profile need be solved iteratively \citep{chavas2016model,chavas2015model} . The empirical models for converting the resulting gradient wind speeds to surface winds are different from the counterparts in the other two types of models. 
\newline

\paragraph{Validation} All three models have been validated by comparing indicators (characteristics) obtained from the simulation results with those estimated from TC in the database. The validation of the models consisted of taking input adopted from one or multiple TCs and determining if the indicators estimated over the output of a model match their target values (at least in the probabilistic sense). The appropriate indicator of validation may vary as it is dependent on the major characteristics of the specific type of models. Results of Type II models have been validated by almost all recent available TCs in the database in terms of the maximum wind speed as well as the time histories of both wind speed and direction \citep{li2016typhoona, vickery2000hurricane, vickery2009hurricane-b}. The validation results suggest the Type II model by Vickery qualifies as a design tool in the ASCE specification \citep{vickery2009hurricane-c}. The indicators of Type I model include the pressure snapshot, vertical profile of mean wind speeds, and the radial profile, suggesting satisfactory validation. The radial profile using the Type III model also matched the target profiles well \citep{chavas2015model,emanuel2004tropical,emanuel2006statistical}. Finally, it is suggested that the CDF of wind speeds generated by all the models should be validated for the Type II and III models \cite[e.g.,][]{emanuel2006statistical, li2016typhoon}. 
\newline

\noindent\textbf{Comparison } \\An investigation benchmarked by the MM5 \citep{liu1997multiscale} suggests that generally the 3D models may outperform the 2D models, underscoring that the nonlinear solution is always superior \citep{kepert2010slab-i,kepert2010slab-ii, kepert2014reply}. It also states that the Type II model is unable to replicate accurately the TC in the database due to the model neglecting many critical factors that are key for accurate results \citep{kepert2010slab-i}. Comparisons have also been carried out between specific models that belong to Type II and Type III \citep{smith2008critique}, or between models that belong to the same type \citep{snaiki2017linear, wills2000review}. Currently, a comprehensive comparison covering all three model types is not available. 
\newline

\paragraph{Boundary layer wind speed profile} While a TC is still over the ocean, the marine wind speed profile in the boundary layer varies with model type. Type I models and height-resolved Type II models can generate the profile of the simulated wind field. Whether the generated profile can approximate well the ones estimated by dropsonde measurements in the TC database is still open for debate \citep{kepert2001dynamics-ii, kepert2011choosing, kepert2013how, montgomery2014comments, smith1968surface}. For the remaining Type II models, an empirical profile formula has been proposed based on extensive statistics over the TC database and applied to the linearized 3D model of Type I \citep{vickery2009hurricane-b}. Although such measurements may occasionally suggest applicability of the power law \citep{song2016characteristics}, this formula is a deeply revised version of the logarithmic law. This profile, developed for the Type II model, may be applicable to the Type III model, but, so far, the latter model simply adopts a constant value of 0.85 to convert wind speeds from the gradient height to 10 m over the ground \citep{chavas2015model}. In contrast to the over-ocean case, after landfall the profile of a TC may be altered as described by the semi-empirical model \citep{snaiki2018semiempirical}. Finally, for a specific land-based site, the wind speed profile of concern is heavily influenced by its surrounding terrain \citep{huang2013prediction}.
\newline

\paragraph{Drag coefficient} The surface drag coefficient is a common parameter shared by all three model types. It can influence significantly the final simulation results, especially the predicted maximum wind speed \citep{li2015observations, powell2003reduced}. Currently, the velocity-dependent and constant models are extensively used for the over-ocean and over-land cases, respectively, but their appropriateness is still arguable \citep{smith2014sensitivity}. 
\newline

\paragraph{Turbulence} Recent empirical data shows no significant difference between gust factors in TC and non-TC winds \citep{vickery2009hurricane-a}. This implies that local terrain dominates turbulence effects even in winds generated by a TC, thus allowing for the use of gust factor models based on regular wind data [e.g., ESDU (1983)]. In contrast, a recent study discovers an apparently different spectral model for the turbulence in TC winds. In this conceptual model, the spectral contents of TC winds at the highly reduced frequency range are higher than non-TC winds (hu2017tropical; Li et al. 2015a).
\newline

\paragraph{Nonstationarity} Typically, TC winds involve both short-term and long-term nonstationary properties of concern in performance-based wind engineering. The mean wind speed, direction, and spectral contents of a TC are all time–dependent, evolving within the lifetime of TC. Considering short-term nonstationarity of winds, the nonstationary wind loading is induced on a target structure, whose effects on structural response as well as performance have been investigated \citep{kareem2018generalized, kwon2009gustfront, yau2011hurricane}. Long-term nonstationarity effects relate to the life-cycle of the target structure. Over the long term, the input of TC models, e.g., the occurrence rate and the intensity, may evolve with time because of climate change \citep{emanuel2005increasing}. These long-term nonstationary effects have been assessed by integrating the TC models with the current climate change models \citep{emanuel2008hurricanes,lauren2014assessing,lin2015integrated,liu2014projections}.

\section{Software and Systems}
\label{sec:storm_wind_tools}

Currently, there is no exclusive software publicly available for generating the wind hazard IM for TC simulations; however, a module designed for such a task is included in both the HAZUS \citep{vickery2006hazusmh} and FCHLPM (Florida Commission on Hurricane Loss Projection Methodology \citep{hamid2010predicting, powell2005state} software. Both programs are based on Type II TC models, although the technical details populating the programs are slightly different. The programs are Windows based, publicly available, and controlled by a GUI. Such in-house software exists in research laboratories around the world.

%\end{document}
