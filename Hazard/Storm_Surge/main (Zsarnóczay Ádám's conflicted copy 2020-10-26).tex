%%%%%%%%%%%%%%%%%%%% author.tex %%%%%%%%%%%%%%%%%%%%%%%%%%%%%%%%%%%
%
% sample root file for your "contribution" to a contributed volume
%
% Use this file as a template for your own input.
%
%%%%%%%%%%%%%%%% Springer %%%%%%%%%%%%%%%%%%%%%%%%%%%%%%%%%%%%%%%%%


%% RECOMMENDED %%%%%%%%%%%%%%%%%%%%%%%%%%%%%%%%%%%%%%%%%%%%%%%%%%%
%\documentclass[graybox]{svmult}
%
%% choose options for [] as required from the list
%% in the Reference Guide
%
%\usepackage{mathptmx}       % selects Times Roman as basic font
%\usepackage{helvet}         % selects Helvetica as sans-serif font
%\usepackage{courier}        % selects Courier as typewriter font
%\usepackage{type1cm}        % activate if the above 3 fonts are
                             % not available on your system
%
%\usepackage{makeidx}         % allows index generation
%\usepackage{graphicx}        % standard LaTeX graphics tool
%                             % when including figure files
%\usepackage{multicol}        % used for the two-column index
%\usepackage[bottom]{footmisc}% places footnotes at page bottom
%
%% see the list of further useful packages
%% in the Reference Guide
%
%\makeindex             % used for the subject index
%                       % please use the style svind.ist with
%                       % your makeindex program
%
%%%%%%%%%%%%%%%%%%%%%%%%%%%%%%%%%%%%%%%%%%%%%%%%%%%%%%%%%%%%%%%%%%%%%%%%%%%%%%%%%%%%%%%%%%
%
%\begin{document}

\title{Tropical Cyclone - Storm Surge}
% Use \titlerunning{Short Title} for an abbreviated version of
% your contribution title if the original one is too long
\author{
    \textbf{Tracy Kijewski-Correa} 
    \and George Deodatis
    \and Yuki Miura}
\tocauthor{}
\authorrunning{Kijewski-Correa et al.}
% Use \authorrunning{Short Title} for an abbreviated version of
% your contribution title if the original one is too long
%\institute{Name of First Author \at Name, Address of Institute, %\email{name@email.address}
%\and Name of Second Author \at Name, Address of Institute %\email{name@email.address}}
%
% Use the package "url.sty" to avoid
% problems with special characters
% used in your e-mail or web address
%
\maketitle

Within the context of the applications envisioned by the SimCenter, coastal simulations must be responsive to a specific tropical or extratropical storm scenario, generally described by a storm track or set of parameters representative of that track, and sensitive to the local topography and offshore bathymetry. These simulations yield geospatially distributed estimates of storm surge (storm-induced rise in seawater levels, primarily caused by wind) for the purposes of direct and indirect loss assessment for coastal communities (jacob2011responding). Estimates of storm-induced inundation, due to combined effects of storm surge and waves driving water over land, are important outputs from any simulation environment that help quantify damage to structures as well as above and below ground civil infrastructure.
In general, it is preferable to manage the model fidelity versus computational efficiency trade off through the use of surrogate models, which can reduce CPU time from hundreds and even thousands of hours to minutes. This enables computationally efficient means to characterize uncertainty in the hazard (e.g., the hurricane track) for the purpose of risk assessment (kijewski-correa2014cybereye).

\section{Common Modeling Approaches}
\label{sec:storm_surge_methods}

This section examines the three classes of models commonly coupled to capture storm surge and accompanying wave effects nearshore and overland, as well as surrogate models that can be tailored to these coupled models for a computationally efficient simulation alternative. Note: this is not an exhaustive presentation of the simulation tools available for coastal hazards but focuses only on those viewed as the industry standard. Simulation tools for coastal hazards have continued to evolve, including the ADCIRC \citep{adcirc_web} and GEOCLAW/CLAWPACK software (mandli2016clawpack) with a geographical information system (GIS).
\newline

\noindent\textbf{Storm surge heights and inundation} \\Numerical models for storm-surge simulations are typically based on single-layer-depth average differential equations describing fluid motion driven by storm winds that make assumptions about the ocean’s response to the storm. This approach is significantly more efficient computationally compared to using a CFD-based approach, such as OpenFOAM. The available numerical models differ in their computational solution strategies, which has implications on the spatial and temporal resolution of the simulations, the required computational resources and runtimes, and the required input data and model parameters. Generally, these models capture the amplitude of long-period, long-gravity waves and do not simulate short-period wave effects, which are addressed in subsequent sections.

The National Weather Service (NWS) utilizes a storm-surge model called Sea, Lake and Overland Surges from Hurricanes (SLOSH), which solves the water equations using local grids (jelesnianski1992NWS48). It was developed to provide real-time estimates of storm surge with computational capabilities of the 1990s; therefore, the grid resolution and the resulting spatial resolution of the results are fairly coarse. As reported by Mandli and Dawson (2014), a primary limitation of SLOSH is “the limited domain size and extents allowed due to the grid mapping used and formulation of the equations.” Nevertheless, since its initial development, SLOSH has continued to be updated and is used for real-time forecasts of surge for public advisories and to inform emergency responders.

The ADvanced CIRCulation (ADCIRC) methodology is commonly regarded as the state-of-the-art in coastal storm-surge simulation (Luettich Jr. et al., 1992), capable of providing significantly more accurate simulations than methods based on SLOSH (Resio and Westerink, 2008) in near-shore coastal regions. As such, ADCIRC is the preferred methodology for coastal storm-surge investigations by the U.S. Army Corps of Engineers (USACE) and in the generation of FEMA Flood Insurance Rate Maps (FIRMSs). ADCIRC solves the equations of motion describing a moving fluid on a rotating earth, formulated using the traditional hydrostatic pressure and Boussinesq approximations, and discretized in space using the finite-element method and in time using the finite-difference method. ADCIRC can be run either as a 2D depth integrated (2DDI) model or as a 3D model, with elevation resulting from the solution of the depth-integrated continuity equation in generalized wave-continuity equation (GWCE) form. Furthermore, velocity is obtained from the solution of either the 2DDI or 3D momentum equations, retaining all nonlinear terms. ADCIRC simulations have been validated for major hurricanes such as Katrina, Ike, Gustav, and Iniki (kennedy2011origin, 2012). 

GEOCLAW, the third computational platform, lies between SLOSH and ADCIRC in terms of modeling resolution and computational cost. Originally developed to simulate tsunami inundation, GEOCLAW has recently been adapted to simulate storm surge (Berger et al., 2011, Mandli et al., 2014). Based on the CLAWPACK software libraries (leveque2002finite), GEOCLAW is an open-source, finite-volume, wave-propagation numerical model used to estimate hurricane-induced storm surge along a coastline. For overland flooding, the model uses Manning's N to parameterize roughness due to objects such as trees and small-scale structures that cannot be resolved computationally. Adaptive mesh refinement allows GEOCLAW to place computational resources where and when they are needed during a simulation. Thus, the overall cost of the simulation is reduced, while retaining the same or similar accuracy characteristics to ADCIRC. Shown in Figure 1 1 is a comparison of results calculated using GEOCLAW versus ADCIRC. 
\newline

\noindent\textbf{Nearshore wave models} \\The three platforms described above simulate the long-wave surge heights but do not capture local wave effects. To overcome this limitation, ADCIRC simulations have been coupled with different nearshore wave models. ADCIRC has been coupled previously with Simulating Waves Nearshore (SWAN) (kennedy2012tropical), a third-generation wave model developed at Delft University of Technology, that computes random, short-crested wind-generated waves in coastal regions and inland waters (Zijlema, 2010). The most recent North Atlantic Coastal Comprehensive Study (NACCS) (USACE 2015) employs STWAVE, which is a steady-state, finite difference spectral model for nearshore wind-wave growth and propagation based on the wave action balance equation (Smith et al., 2001). STWAVE simulates depth-induced wave refraction and shoaling, current-induced refraction and shoaling, depth- and steepness-induced wave breaking, diffraction, wave growth because of wind input, and wave–wave interaction and white capping that redistributes and dissipates energy in a growing wave field. Figure 1 2 validates the coupled hydrodynamic models used in the NACCS by comparing to measurements across historical storms or tide predictions (Nadal-Carabbalo et al., 2015). 
\newline

\noindent\textbf{Wave run up overland} \\Even when coupled with an appropriate nearshore wave model, ADCIRC simulates only the storm-surge elevation and not the additional impact of wave run up, which is particularly important for predicting losses to buildings and infrastructure in a storm event. Supplementary wave run-up simulations are required to capture the interaction of the waves with the shoreline and any coastal protective features along coastal transects. Wave run-up calculations are executed at transect locations generally selected by segmenting the defined coastline in the areas of interest and selecting the transect density proportional to computational demand. Each transect is then discretized to capture the site-specific bathymetry (offshore) and topography (onshore) along its length. Moreover, transects must accurately capture the current condition of coastal protective features, e.g., dunes, in order to effectively predict the total run up inland. Inputs from the ADCIRC+STWAVE model are fed into a one-dimensional (1D) Boussinesq model executed at the pre-selected transects in order to estimate the wave run up overland \citep{demirbilek_2009}. 
\newline

\noindent\textbf{Surrogate modeling approach} \\Given the high degree of sophistication and computational resources required to execute just one high-fidelity simulation (e.g., an ADCIRC+STWAVE/SWAN run), alternative simulation tools have been developed recently to enable a wider range of users to employ these models for hazard characterization, risk assessment, and design of coastal protective strategies. Most notably, surrogate modeling approaches can efficiently evaluate hurricane wave and surge responses by leveraging databases of existing high-fidelity simulations normally driven by a collection of historical and synthetic hurricane tracks (USACE, 2015). This is made possible by formulating a simplified description of a storm scenario by a small number of model parameters corresponding to its characteristics at landfall. The scenarios in the database are then parameterized with respect to this model parameter vector and ultimately provide an input–output dataset. Because the geospatial representation often covers a regional coastline (typically represented by a large number of nodes) and resolves the coastal hazards at different times during the hurricane’s history, the dataset is often high-dimensional. After correcting for any dry nodes at inland locations, the surrogate model is then built to approximate this input-output relationship.

Although the initial implementations of the surrogate modeling approach relied upon a moving least-squares-response surface methodology, more recent implementations for natural hazard risk assessment now employ a Kriging metamodel for this purpose (jia2013kriging). To further reduce the computational burden pertaining to both speed of execution and more importantly memory requirements, this approach is coupled with principal component analysis (PCA) as a dimensional reduction technique. The metamodel is then developed in this low-dimensional latent space (in this case below 100), with the predictions transformed back to the original space for visualization purposes. This PCA implementation contributes to very large computational savings necessary to enable the evaluation of a large ensemble of scenarios as required for a probabilistic evaluation while circumventing the need for HPC resources (jia2013kriging). Validation of these surrogate models using leave-one-out cross validation (Taflanidis et al., 2017) suggested high accuracy, with coefficient of determination close to 0.96 and a correlation coefficient close to 98\%. By permitting rapid evaluation of alternate storm scenarios, surrogate models offer an effective way to communicate simulation results to urban planners and emergency managers. One such implementation is a software system developed to assess storm surge risks on the coast of New Jersey (NJcoast, 2018a).

\section{Required Inputs and Resulting Outputs}
\label{sec:storm_surge_io}

High-fidelity computational simulations of coastal hazards require: (1) storm track information, including the relevant description of the hurricane wind field to drive the model; (2) the topography and bathymetry along the coastline; and (3) the land use/land cover data for the simulation of wave run up on shore. The simulations are inherently sensitive to assumptions made regarding tides at the time of landfall. The coupling of a storm surge + nearshore wave + wave run-up model will yield geospatially-distributed, time-dependent responses, i.e., the mean water elevation, max water elevation, max water depth, and significant wave height (or limit of moderate wave action). Such responses can be generated either by the coupling of the aforementioned high-fidelity models or a surrogate model tuned to a database of results from these models. A brief summary of specific inputs required for ADCIRC, the wave run-up models, and the related surrogate models are as follows:
\newline

\noindent\textbf{ADCIRC Inputs/Forcing } \\ADCIRC requires both boundary conditions, as well as forcing as inputs to the simulation. Boundary conditions include: 
\begin{itemize}
    \item specified elevation (harmonic tidal constituents or time series)
    \item specified normal flow (harmonic tidal constituents or time series)
    \item zero normal flow
    \item slip or no slip conditions for velocity
    \item external barrier overflow out of the domain
    \item internal barrier overflow between sections of the domain
    \item surface stress (wind and/or wave radiation stress)
    \item atmospheric pressure and outward radiation of waves (Sommerfeld condition)
\end{itemize}
	
\noindent ADCIRC can be forced by 
\begin{itemize}
    \item elevation boundary conditions
    \item normal flow boundary conditions
    \item surface stress boundary conditions
    \item tidal potential
    \item earth load/self-attraction tide
\end{itemize}

In the case of a comprehensive evaluation of coastal hazards due to hurricanes and nor’easters, planetary boundary layer numerical models are used to generate wind and pressure fields that drive these high-fidelity storm surge and wave hydrodynamic models (see Section 1.5 for details). These wind and pressure fields are developed for a suite of simulated or historical storm tracks for the targeted region (anywhere from hundreds to even thousands of storm tracks).
\newline

\noindent\textbf{Wave run-up inputs} \\In addition to the topography and bathymetry data at each identified transect, the wave run-up model must receive inputs from the coupled storm surge model, e.g., ADCIRC+STWAVE. As selected transects may not align with saved data points from the ADCIRC+STWAVE simulations, a nearest neighbor approach is required to identify the inputs to the wave run-up models, specifically: the peak wave period from STWAVE, zero moment wave height from STWAVE, and water elevation from the closest ADCIRC data point. 
\newline

\noindent\textbf{GEOCLAW inputs} \\GEOCLAW requires gauge and topographical data as inputs to compute depth and momentum of the water at a number of locations. The storm surge simulation is driven by the provided time-dependent wind field and pressure distribution. The full field is constructed using the equations by Holland (1980).
\newline

\noindent\textbf{Surrogate model inputs} \\Inputs to the surrogate model are twofold: the primary input required to develop the surrogate model itself is the aforementioned database of high-fidelity simulations for a family of storm tracks that may include tropical and extra-tropical storms. Once developed, users of the surrogate model input only a collection of parameters necessary to describe the storm scenario based on its characteristics at landfall: 

\begin{itemize}
    \item reference location (latitude, longitude)
    \item track heading (angle)
    \item central pressure (or pressure difference)
    \item forward speed
    \item radius of maximum winds
\end{itemize}

More recently, this implementation was further simplified to enable simulation based on only reference location and storm strength (Category 1-5) (NJcoast, 2018a). It is important to emphasize that once the surrogate model is tuned to high-fidelity simulation data for a specific geographic location, it can efficiently provide predictions for storm scenarios of varying characteristics, even if that scenario does not match any of those within the original database of high-fidelity simulations. 

\section{Primary Software Environments}
\label{sec:storm_surge_tools}

The execution environments are briefly summarized below. 
\newline

\noindent\textbf{ADCIRC and coupled models} \\ADCIRC has been optimized by unrolling loops for enhanced performance on multiple computer architectures and can be executed on any operating system with a working FORTRAN compiler. These include large commercial Unix systems (IBM Power \& Blue Gene, Cray, SGI, and Sun), Linux- and FreeBSD-based clusters, and personal workstations running Windows or Mac OSX. ADCIRC includes MPI library calls to allow it to operate at high efficiency on parallel computer architectures, which is often preferable for simulations over large domains where a single hurricane realization can require thousands of CPU hours. Coupled ADCIRC+SWAN models are available on all of the aforementioned platforms (with the exception of Windows), while the coupled ADCIRC+STWAVE model is available on all the platforms including Windows PCs as part of the Coastal Storm Modeling System (CSTORM-MS). ADCIRC and its parallel implementation, PADCIRC, along with the coupled ADCIRC+SWAN software, are available on DesignSafe.
\newline

\noindent\textbf{CLAWPACK/GEOCLAW} \\Clawpack (“Conservation Laws Package”) is a collection of finite-volume methods for linear and nonlinear hyperbolic systems of conservation laws. Clawpack employs high-resolution Godunov-type methods with limiters in a general framework applicable to many kinds of waves. GEOCLAW is an open-source, finite-volume, wave-propagation software, which is implemented in CLAWPACK, to estimate hurricane-induced storm surge with adaptive mesh refinement. The CLAWPACK 5.4.0 suite and the GEOCLAW tools are available through DesignSafe.
\newline

\noindent\textbf{SLOSH} \\SLOSH (Sea, Lake and Overland Surges from Hurricanes) is a computerized numerical model developed by the National Weather Service (NWS) to estimate storm-surge heights determined from historical, hypothetical, or predicted hurricanes by taking into account the atmospheric pressure, size, forward speed, and track data. These parameters are used to create a model of the wind field that drives the storm surge. The SLOSH model consists of a set of physics equations that are applied to a specific locale's shoreline to incorporate the unique bay and river configurations, water depths, bridges, roads, levees, and other physical features. Storm-surge forecasts developed using SLOSH are available at https://www.nhc.noaa.gov/surge/slosh.php.

\section{Major Research Gaps}
\label{sec:storm_surge_gaps}

Given the sophistication and computational demands of high-fidelity models like ADCIRC and GEOCLAW, continued advancements in metamodeling and other similar approaches will ensure that the wider research community can engage with computational simulation tools for coastal hazards without the barrier to entry that currently exists. Note: whether employing these high-fidelity models or a companion surrogate model, the resulting time-evolving water depth and velocity must translate into loadings on buildings and infrastructure. In this regard, these models face similar limitations as wind-field models given the complexity of interactions with their surroundings. Accurately capturing the physics of the flow overland and the effect of its interaction with the built environment on the load description remains a challenging problem, even without further accounting for the effects of debris transported in the flow. Identifying means to reasonably determine the impact of these interactions on the load description—without having to support an intensive CFD investigation—will enable a wider range of researchers to evaluate the impacts of coastal hazards. 

%\end{document}
