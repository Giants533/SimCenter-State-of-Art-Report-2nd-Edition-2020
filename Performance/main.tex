
\begin{partbacktext}
\part{Performance Assessment}\label{part:Performance}

The built environment is a collection of various types of assets that affect the well-being and quality of life of residents in an urban area. The list of such assets includes residential and commercial buildings, bridges, networks of roads, railways, pipelines and power lines, and their supporting facilities. This part of the report focuses on evaluation of the damage and direct consequences that typically describe the immediate impact of the disaster and assigns a performance measure to the various assets that make up the built environment. Part \ref{part:recovery} focuses on the simulation of recovery and consequences that consider interdependencies between infrastructure systems, the built environment, and local communities.

The performance of assets heavily depends on the description of the hazard and the simulation of asset response to one or more characteristic events that are consistent with the hazard at the asset location. Although this chapter focuses on performance assessment, some of the tools listed here have hazard and response estimation capabilities as well. Those features have been covered in Parts \ref{part:hazard} and \ref{part:response}, respectively. 

Seismic performance assessment of buildings has received a lot of attention from the research community and funding agencies in the past few decades \citep{atc1985atc13, fema1997guidelines, fajfar2004performancebased, kircher2006hazus}. Consequently, the most sophisticated and mature methods are available in that area \citep{atc2012p-58}. Several researchers have focused on adopting these methods for other asset types \citep{werner2006redars, chmielewski2016response} and for other types of hazards \citep{vickery2006hazus, bernardini2015performance, attary2017performancebased, barbato2013performancebased, lange2014application}. Although this led to similar methods being used for various types of assets, some of the synergies between these methods are not yet utilized in research.

The integration of multiple hazards and multiple asset types in a simulation framework is an important step towards a comprehensive performance assessment for the built environment. It is important to recognize the difference between multiple hazard studies and multi-hazard studies. Following the naming convention suggested by \citet{bruneau2017state}, the former studies consider multiple, independent hazards in an area, while the latter studies consider the interactions and cascading effects among those hazards as well. In a multi-hazard analysis, the majority of the resulting risk is not from concurrent extreme events because the probability of two of such events happening simultaneously is very low. The characterization and modeling of more frequent hazards becomes more important and influential to the results. There are already several examples of multi-hazard studies in the literature for a wide range of hazard and asset types: earthquake mainshock and aftershock (e.g., \cite{nazari2015effect, zhang2013damage}); ground shaking, liquefaction, and landslides (e.g., \cite{elgamal2008three, kojima2014large}); earthquake and tsunami (e.g., \cite{akiyama2014reliability, carey2019multihazard}); hurricane wind, windborne debris, storm surge (e.g., \cite{lin2010windborne, park2014abv}), and rainwater (e.g., \cite{pita2012assessment}); and flood and sea level rise (e.g., \cite{hinkel2014coastal}). It is important to modeling interactions not only at the hazard level (e.g., \cite{gill2014reviewing}, but also include site effects, disruptions of systems as well as system-level social and economic consequences. The analyses that model interactions beyond the hazard level are referred to as Level II interactions by \citet{zaghi2016establishing}.

This part is organized around the types of assets and asset-networks needed to arrive at a comprehensive description of the performance of an urban region under natural hazards. The content is divided into four chapters to facilitate navigation, but the importance of an integrated assessment is emphasized by highlighting the synergies across assets and hazards throughout.

\end{partbacktext}
