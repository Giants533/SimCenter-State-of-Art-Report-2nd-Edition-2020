%%%%%%%%%%%%%%%%%%%% author.tex %%%%%%%%%%%%%%%%%%%%%%%%%%%%%%%%%%%
%
% sample root file for your "contribution" to a contributed volume
%
% Use this file as a template for your own input.
%
%%%%%%%%%%%%%%%% Springer %%%%%%%%%%%%%%%%%%%%%%%%%%%%%%%%%%%%%%%%%


%% RECOMMENDED %%%%%%%%%%%%%%%%%%%%%%%%%%%%%%%%%%%%%%%%%%%%%%%%%%%
%\documentclass[graybox]{svmult}
%
%% choose options for [] as required from the list
%% in the Reference Guide
%
%\usepackage{mathptmx}       % selects Times Roman as basic font
%\usepackage{helvet}         % selects Helvetica as sans-serif font
%\usepackage{courier}        % selects Courier as typewriter font
%\usepackage{type1cm}        % activate if the above 3 fonts are
                             % not available on your system
%
%\usepackage{makeidx}         % allows index generation
%\usepackage{graphicx}        % standard LaTeX graphics tool
%                             % when including figure files
%\usepackage{multicol}        % used for the two-column index
%\usepackage[bottom]{footmisc}% places footnotes at page bottom
%
%% see the list of further useful packages
%% in the Reference Guide
%
%\makeindex             % used for the subject index
%                       % please use the style svind.ist with
%                       % your makeindex program
%
%%%%%%%%%%%%%%%%%%%%%%%%%%%%%%%%%%%%%%%%%%%%%%%%%%%%%%%%%%%%%%%%%%%%%%%%%%%%%%%%%%%%%%%%%%
%
%\begin{document}

\title{Buildings}
% Use \titlerunning{Short Title} for an abbreviated version of
% your contribution title if the original one is too long
\author{
    \textbf{Adam Zsarnóczay}
    \and Jack W. Baker}
\tocauthor{}
\authorrunning{Zsarnóczay and Baker}
% Use \authorrunning{Short Title} for an abbreviated version of
% your contribution title if the original one is too long
%\institute{Name of First Author \at Name, Address of Institute, %\email{name@email.address}
%\and Name of Second Author \at Name, Address of Institute %\email{name@email.address}}
%
% Use the package "url.sty" to avoid
% problems with special characters
% used in your e-mail or web address
%
\maketitle

Buildings are arguably the most important asset type when it comes to direct consequences of a natural disaster. A severely damaged or collapsed building may result in loss of life, injuries, and significant capital losses. Community disruption and indirect consequences are heavily affected by damage to transportation infrastructure and lifelines.

Conceptually, building performance assessment has been moving from a holistic towards an atomistic approach. Instead of trying to characterize building damage as a whole, buildings are disaggregated into sets of structural components, non-structural components, and contents (Figure 3 1). Component damage is estimated based on the response of the building to the natural disaster. The information about component damages at various locations in the building allows better understanding of the consequences of the disaster. 

Although these sophisticated models promise more information about building performance, their veracity demands more detailed input data about the building and its components. Consideration of the uncertainty that stems from the limited amount of building information available is essential for a robust performance evaluation \citep{bradley2013critical}. The methods available for quantification and propagation of such uncertainty are discussed in Chapter 4.
 
\section{Input and Output Data}
\label{sec:perf_bldg_io}

The following types of data are required to evaluate the performance of a building:
% \newline

\paragraph{Hazard characterization} When the building performance is not conditioned on a particular disaster scenario, but it is evaluated considering all possible scenarios within a time period, we need to estimate the likelihood of each possible scenario. The hazard curve describes the rate of exceeding various levels of an Intensity Measure (IM) over the time period of interest. More information about the description of the hazard is provided in Chapter 1.
\newline

\paragraph{Engineering demand parameters (EDPs)} Modern building performance assessment uses EDPs as proxies for the detailed history of building response under a natural disaster event. EDPs shall have high correlation with the building damage of interest, and they shall be estimable with sufficiently high accuracy through numerical analysis. Estimation of EDPs first requires a building response model that is typically created in one of the environments listed in Section 2.1. Second, the building model needs to be excited with loads that correspond to a particular hazard event. The inputs required for these models and analysis were described in previous chapters.

EDPs are extracted from the structural response history during post-processing. Seismic performance assessment often uses peak responses at every story such as peak story drift ratios and peak floor accelerations (FEMA, 2012). Because other types of hazards result in different response and damage, they are characterized by other types of EDPs, such as maximum inundation depth under a tsunami (Reese, 2011). 
% \newline

\paragraph{Component characteristics} Depending on the complexity of the performance assessment method and data availability, buildings are described as a system of components or component-groups. For example, the component-group-based approach followed by HAZUS (FEMA Mitigation Division, 2018b) aggregates structural components, non-structural components, and contents into three groups. The FEMA P58 method represents the other end of the spectrum; it disaggregates the building into units of components with identical behavior. The component-group-based method typically requires rather generic inputs such as the type of structural system and the occupancy type to infer component behavior. The more detailed methods use the quantity, direction, and location of each component unit on each floor of the building to estimate their damage.
% \newline

\paragraph{Fragility functions} The fragility functions describe the likelihood of exceeding a particular Damage State (DS) as a function of EDP magnitude. Component damage is classified into a finite number of DSs, so that each DS groups damage scenarios with similar consequences. Fragility curves are essential for every loss assessment. A large part of fragility data is proprietary, especially in the field of wind and water hazards. HAZUS provides fragility functions for component-groups for various hazards. FEMA P58 enables more sophisticated analysis for seismic hazards by providing a database with detailed description of more than 700 types of components. 
% \newline

\paragraph{Consequence functions} Direct consequences of building damage are quantified by various types of consequence functions. Each DS has its corresponding set of consequence functions. These functions are defined by additional input data such as repair cost per component unit, affected area for calculation of injuries, or the probability that a component in that particular DS would trigger an unsafe placard for the building. Similar to the other inputs above, these data are not exact, and the description and propagation of their uncertainty is an important part of the calculation method.

Indirect consequences and the influence of building damage on surrounding buildings and infrastructure have been considerably more difficult to model because of the scarcity of data that could be used for model calibration. Decision variables in this group include non-immediate injuries and hospital demand, displaced households and short-term shelter needs, business interruption costs, demand surge, and its influence on reconstruction cost and downtime estimates \citep{arup2013resiliencebased}.
% \newline

\paragraph{Decision variables (DVs)} Performance assessment is typically executed in a stochastic framework; the DVs are considered random, and the raw results of the assessment are at least thousands but often hundreds of thousands of samples of each variable. Therefore, interpretation and visualization of the results is an important part of the process. The majority of applications focus on mean or median values to describe central tendencies with the 10th and 90th percentiles used to illustrate the variability of results. High-performance computing and the improvement in the quality of input data create the incentive to improve estimates of the tails of the distributions and to look at the joint distribution of the variables. These analyses reveal details of complex systems that are often overlooked when focusing only at central tendencies.

\section{Modeling Approaches}
\label{sec:perf_bldg_methods}

The main assumption of the stochastic model for building performance assessment is that the uncertainty in the DVs can be estimated through the following series of independent calculations: 

\begin{itemize}
    \item describe a set of IM levels (e.g., spectral acceleration intensities) and corresponding likelihoods based on the hazard at the building location over a given time period
    \item describe the building response through EDPs given the level of the IM
    \item describe component or component-group DSs given a set of EDP realizations
    \item describe consequences using DVs given the DS of each component or component-group
    \item aggregate DVs from all components or component-groups
\end{itemize}

The calculations can be performed independently if the models used for these calculations are decoupled (e.g., the DS for two different IM levels is assumed identical if they result in identical EDPs).

The above methodology has been developed to quantify the seismic performance of buildings. It is the basis of the widely used HAZUS Earthquake Model, and it led to the development of the probabilistic seismic performance assessment framework in the Pacific Earthquake Engineering Research (PEER) Center \citep{porter2001assemblybased}. That framework, and the often cited ``triple integral,'' was the foundation of the FEMA P58 document that is considered the state-of-the-art method for seismic performance assessment of buildings today.

When it comes to building performance assessment under non-seismic hazards, the models and methods are typically more approximate. This partly stems from the lack of publicly available high-quality databases--which would drive more sophisticated model development--and from the different nature of the problem. The impact and disruption of earthquakes and hurricanes are very different in both spatial and temporal distribution, with hurricanes having a severe impact on a larger region over a longer time period. Therefore, the focus for hurricane and flood models have always tended to be more regional where capturing the detailed response of individual structures receives less attention.

\section{Research Gaps and Needs}
\label{sec:perf_bldg_gaps}

Tom O'Rourke: there is a need to better understand urban seismic vulnerability, the interaction between buildings and also between buildings and the other components of the built environment.

\section{Software and Systems}
\label{sec:perf_bldg_tools}

The following is a list of software that provides features required for state-of-the-art research in building performance assessment:
% \newline

\paragraph{CAPRA} Development of the Comprehensive Approach to Probabilistic Risk Assessment (CAPRA) platform was initially supported by the World Bank and the Inter-American Development Bank; it has been managed by Uniandes (Universidad de los Andes in Colombia) since 2017. CAPRA is designed to become a multi-hazard framework based on several modules that handle different tasks of the risk assessment workflow. The currently available modules allow risk assessment using vulnerability functions for several types of hazards (e.g., earthquake, hurricane, and flood). The open source CAPRA framework uses Visual Basic .NET and provides applications in a Windows environment.
% \newline

\paragraph{MAEViz} Developed by the Mid-America Earthquake Center (MAE), 
MAEViz is based on the HAZUS methodology for scenario risk assessment and it allows users to write their own extensions. Through these added modules, its functionality is not limited to building performance assessment and allows analysis of infrastructure and lifeline performance as well as indirect consequences in the region. It uses a Windows-based application with a user interface to guide the user through the analysis. MAEViz is open source and has been integrated into several platforms in the US [e.g., ERGO, mHARP] and in Europe [e.g., SYNER-G \citep{pitilakis2014synerg} and HAZturk \citep{karaman2008earthquake}].

\paragraph{HAZUS 4.2} The FEMA-supported HAZUS tool was already introduced in Section 1.1. The damage and loss assessment modules in HAZUS use the component-group-based approach and categorize components into structural, non-structural, and content groups. The methodology provides estimates of direct and indirect consequences of damage. Its efficiency allows it to be scaled to a regional level without having to resort to HPC.
% \newline

\paragraph{OpenQuake} The GEM Foundation develops and maintains this tool. The source code is written in Python, open source, and publicly available at a github repository. OpenQuake provides a platform to perform regional disaster risk assessment. The Hazard part of the library has already been mentioned in Section 1.1. The Risk part of the library performs a component-group based performance assessment that is similar to the approach by HAZUS. Input data for the platform is collected and made publicly available in an online repository at platform.openquake.org. Currently, OpenQuake leans heavily towards seismic hazard and risk assessment, but there are developments towards flood impacts, and the framework is sufficiently flexible to allow other extensions as well.
% \newline

\paragraph{OpenSLAT} The Open Seismic Loss Assessment Tool is an open-source library developed at the University of Canterbury written in C++ and Python. It is publicly available and allows researchers to use the developed functions in their preferred environment. It implements the Magnitude-oriented Adaptive Quadrature (MAQ) algorithm developed by \cite{bradley2010efficient} to efficiently solve the integrals involved in PBE calculations.
% \newline

\paragraph{PACT} The Performance Assessment Calculation Tool developed by the Applied Technology Council (ATC) is a publicly available software that implements the performance assessment methodology in the FEMA P58 document. It is designed to describe the performance of a single building, not a region with a collection of buildings. The software is controlled by a GUI and is available for the Windows platform only. It does not perform hazard and structural response calculations, but rather requires the results of those calculations as inputs. All fragility and consequence functions developed in the FEMA P58 project are conveniently available in PACT.
% \newline

\paragraph{PBE Application} The Performance Based Engineering Application has been developed by the NHERI SimCenter to provide a convenient GUI-based tool for researchers interested in performance assessment \citep{mckenna2018performance}. The GUI provides access to the versatile PBE workflow developed at the SimCenter and allows users to choose the tools and methods they wish to use for hazard estimation, response simulation, and loss assessment. The application facilitates the use of high-performance computing resources by providing a built-in connection to the Stampede 2 supercomputer at UT Austin through DesignSafe \citep{rathje2017designsafe}.

Currently, the application is limited to seismic hazards, with wind and water hazards features under development. Seismic hazard assessment uses OpenSHA (see Section 1.1), response estimation uses OpenSEES (see Section 2.1), and loss assessment uses PELICUN (see below) to perform the calculations. Future versions will expand the set of available tools in the application.
% \newline

\paragraph{PELICUN} The Probabilistic Estimation of Losses, Injuries, and Community resilience Under Natural Disasters is an open-source Python library developed by the SimCenter. It is publicly available at the SimCenter github repository (Zsarnóczay, 2018). The library is designed to provide a versatile, platform-independent and transparent loss-assessment tool for the research community. It is based on a stochastic loss model that allows detailed component-based as well as simplified component-group-based loss assessment. The current version implements the scenario-based assessment from the FEMA P58 methodology. The HAZUS methodology for earthquakes and the time-based assessment option are under development. The library allows researchers to work in their preferred environment and call its functions to perform loss assessment. PELICUN is used in the applications developed by the SimCenter for performance assessment.
% \newline

\paragraph{SP3} The Seismic Performance Prediction Program (SP3) is proprietary software developed by the Haselton Baker Risk Group. It is widely considered the most reliable implementation of the FEMA P58 methodology and ARUP's REDi framework for downtime estimation; it is used by practitioners and researchers. Besides the high-quality implementation, the tool is also bundled with valuable data that facilitates building response estimation, and damage and loss assessment. The tool can be accessed through a web-based interface that guides the user through the steps of performance assessment. Researchers with programming skills can use it in batch mode that enables more powerful analyses. The calculations run on Amazon EC2 servers, which allow users to run complex, demanding analyses within a reasonable timeframe.
