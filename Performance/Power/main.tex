%%%%%%%%%%%%%%%%%%%% author.tex %%%%%%%%%%%%%%%%%%%%%%%%%%%%%%%%%%%
%
% sample root file for your "contribution" to a contributed volume
%
% Use this file as a template for your own input.
%
%%%%%%%%%%%%%%%% Springer %%%%%%%%%%%%%%%%%%%%%%%%%%%%%%%%%%%%%%%%%


%% RECOMMENDED %%%%%%%%%%%%%%%%%%%%%%%%%%%%%%%%%%%%%%%%%%%%%%%%%%%
%\documentclass[graybox]{svmult}
%
%% choose options for [] as required from the list
%% in the Reference Guide
%
%\usepackage{mathptmx}       % selects Times Roman as basic font
%\usepackage{helvet}         % selects Helvetica as sans-serif font
%\usepackage{courier}        % selects Courier as typewriter font
%\usepackage{type1cm}        % activate if the above 3 fonts are
                             % not available on your system
%
%\usepackage{makeidx}         % allows index generation
%\usepackage{graphicx}        % standard LaTeX graphics tool
%                             % when including figure files
%\usepackage{multicol}        % used for the two-column index
%\usepackage[bottom]{footmisc}% places footnotes at page bottom
%
%% see the list of further useful packages
%% in the Reference Guide
%
%\makeindex             % used for the subject index
%                       % please use the style svind.ist with
%                       % your makeindex program
%
%%%%%%%%%%%%%%%%%%%%%%%%%%%%%%%%%%%%%%%%%%%%%%%%%%%%%%%%%%%%%%%%%%%%%%%%%%%%%%%%%%%%%%%%%%
%
%\begin{document}

\title{Electrical Transmission Substations and Lines}
% Use \titlerunning{Short Title} for an abbreviated version of
% your contribution title if the original one is too long
\author{
    \textbf{Iris Tien}
    \and {Craig Davis}
    \and {Thomas O'Rourke}
    \and {Adam Zsarnóczay}}
\tocauthor{}
\authorrunning{Tien et al.}
% Use \authorrunning{Short Title} for an abbreviated version of
% your contribution title if the original one is too long
%\institute{Name of First Author \at Name, Address of Institute, %\email{name@email.address}
%\and Name of Second Author \at Name, Address of Institute %\email{name@email.address}}
%
% Use the package "url.sty" to avoid
% problems with special characters
% used in your e-mail or web address
%
\maketitle

The objective of the simulation of electrical substations and transmission and distribution lines is similar to that of other lifelines. Namely, the objective is to assess the ability to provide critical electricity services for populations under varying scenarios. As relevant for natural hazards engineering, the purpose is to assess the state of these lifeline systems under hazard events and inform decision making on approaches to improve the expected performance of these systems under hazards. Compared to hydraulic or pressure flow analyses for water, sewer, and gas pipeline simulation, the underlying physics of the simulation relies on power voltage flow analysis in addition to system-level analyses that can be conducted through the use of network graphs, for example. Electrical power network are also composed of connected regional and local systems. Local distribution systems are dependent upon the regional grid for supply, requiring multiple levels of analysis to capture the performance of the regional grid and local distribution systems.
 
\section{Input and Output Data}
\label{sec:perf_power_io}

Modeling electrical networks requires information about generation facilities, substations, and transmission and distribution circuits. In addition to the geographical location, the level of voltage is an important property. The availability of electrical power system data is subject to many of the constraints of lifeline data, with privacy and national security concerns and the fact that the databases are privately owned. When there is a lack of network data availability, particularly a lack of detailed information at the distribution level, only low resolution analyses at the transmission system level are possible. Depending on the type of hazard, other details can be required, such as anchorage of components for seismic analysis or elevation information for flood hazard assessment. The required inputs and produced outputs for the power grid depend in part on the type of hazard, e.g., climatic compared to geologic, and the target fidelity for reliability and risk analysis. 

\section{Modeling Approaches}
\label{sec:perf_power_methods}

In general, given the IM at the site, component-specific fragility curves can be used to estimate the damage to a given component under a hazard. For facilities and building-like structures the damage evaluation is similar to the HAZUS method described in Section 3.1. Given the damage to the network component, the direct consequences are evaluated using empirical relationships based on past experience. Direct consequences are typically limited to restoration time and replacement cost in HAZUS. In HAZUS, fragility of electrical substations and distribution circuits is defined with respect to the percentage of subcomponents being damaged.

Modeling indirect consequences of damage in the electric power network is an ongoing area of research \citep{moore2005using}. In HAZUS, interaction between nodes of the power network is not considered. Interaction between other lifeline systems is considered with the following approach: the loss of electric power is assumed to influence the slight/minor and moderate DSs of components in other lifelines that depend on power. More severe DSs are not influenced by the lack of power. The substation that serves connected components is assumed to experience the event at the location of the served component. An even more simplified approach uses a generic damage algorithm to describe the availability of power as a function of an IM. 

\section{Software Systems}
\label{sec:perf_power_tools}

\noindent\textbf{HAZUS 4.2} \\As mentioned in Section 3.3, HAZUS groups electric power and communication networks with the other lifelines and provides similar methodologies for their analysis. Electrical networks are the only lifeline type that has influence on other lifelines in the HAZUS methodology (i.e., damage to the electric network and the consequent loss of power results in loss of function and potential damage in other lifelines). HAZUS 4.2 software allows estimation of lifeline damage and consequent reduction in performance. Further details of the software and its limitations are explained in Section 1.1, and its application to lifeline simulation and pipelines in particular are described in Section 3.3.
\newline

\noindent\textbf{OpenDSS} \\OpenDSS is a software package developed by the Electric Power Research Institute (EPRI). It conducts electrical power system simulation for electric utility power distribution systems. It supports simulations of power flow across frequencies and is mainly used to evaluate distributed energy resource generation, its integration with utility distribution systems, and grid modernization technologies. Assessments do not include the impacts of natural hazards on the system components or network performance for natural hazards engineering applications.
\newline

\noindent\textbf{MatPower} \\MatPower is a free toolbox available in MATLAB for conducting power flow analysis. It does not, however, have the capability to directly model natural hazard effects.

\section{Research Gaps and Needs}
\label{sec:perf_power_gaps}

Similar to the description of the research gaps and needs in water, sewer, and gas pipeline simulation (Section 3.3.4), there is the need for improved consideration of interdependencies between lifelines as related to electrical systems. Given the criticality of electricity for many lifelines, this will enable researchers to better understand the impacts of natural hazards on communities. Advancing component-level fragility curves and improved simulation of structural response for critical facilities will lead to a better estimate of expected damage, both in terms of accuracy and resolution in a simulation. Needed are improved fragility functions for specialized structures for each lifeline sector, e.g., generation plants, substations and components, and transformers for electrical systems. Analysis of equipment supporting these specialized facilities is also needed. The focus on selected structural components in electrical power network damage analysis, where fragility functions do exist, can also lead to missing failure mechanisms and inaccurate risk assessment. Finally, improved simulation of the recovery process will enable researchers to better understand performance and recovery of lifeline services during and after disasters. The high computational cost and sometimes computational intractability of network-level performance analyses for electrical systems is an ongoing research challenge.

\section{References}
