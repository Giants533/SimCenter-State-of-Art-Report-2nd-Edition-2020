
\begin{partbacktext}
\part{Hazard Characterization}

Characterization of natural hazards for engineering applications aims to quantify the severity of the hazard at a particular location or over a pre-defined region of interest. Quantification is often performed through so-called Intensity Measures (IMs) that describe the hazard with one or a few parameters. Peak ground acceleration, permanent ground deformation, average one-minute wind speed, and peak inundation depth are a few examples of such IMs for various natural disasters. The limited number of parameters allows the development of a stochastic hazard model at the site(s) of interest and propagation of uncertainty in the hazard through engineering analyses. 

When structural response is estimated through simulation of the response history of a numeri-cal structural model, the hazard is typically represented by a time-dependent load function. The acceleration time history of a ground motion is an example of such a load function, which is often used for seismic response estimation. These load functions are either selected from historical data (e.g., ground-motion records) or generated based on the stochastic description of local IMs (e.g., local wind inflow conditions for a CFD simulation). The procedures and best practices available for this task will be discussed for each natural hazard below.

Three different types of natural disasters are examined in the following sections: earthquake, hurricane, and tsunami. Some of them present several fundamentally different threats to the built environment, such as ground shaking and liquefaction under earthquakes, or wind and storm surge under hurricanes. These are discussed separately below. 

\end{partbacktext}