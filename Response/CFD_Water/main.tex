%%%%%%%%%%%%%%%%%%%% author.tex %%%%%%%%%%%%%%%%%%%%%%%%%%%%%%%%%%%
%
% sample root file for your "contribution" to a contributed volume
%
% Use this file as a template for your own input.
%
%%%%%%%%%%%%%%%% Springer %%%%%%%%%%%%%%%%%%%%%%%%%%%%%%%%%%%%%%%%%


\title{Computational Fluid Dynamics - Water}
% Use \titlerunning{Short Title} for an abbreviated version of
% your contribution title if the original one is too long
\author{
    \textbf{Michael Motley}
    \and Arindam Chowdhuri
    \and Catherine Gorlé
    \and Ajay Harish
    \and Andrew Kennedy
    \and Seung Jae Lee
    \and Patrick J. Lynett}
\tocauthor{}
\authorrunning{Motley et al.}
% Use \authorrunning{Short Title} for an abbreviated version of
% your contribution title if the original one is too long
%\institute{Name of First Author \at Name, Address of Institute, %\email{name@email.address}
%\and Name of Second Author \at Name, Address of Institute %\email{name@email.address}}
%
% Use the package "url.sty" to avoid
% problems with special characters
% used in your e-mail or web address
%
\maketitle

Computational Fluid Dynamics (CFD) uses numerical methods to solve governing equations that arise in fluid mechanics. While the previous section focused on gaseous fluids (e.g., air), here we focus on applications of CFD for liquid fluids (e.g., water), although modeling of the water's free surface arises in many situations, especially those around natural hazard modeling as it requires modelling the air, the fluid, and the interface between the two. For water, the standard governing equations are the incompressible Navier-Stokes equations (NS equations, \citep{Darrigol2005navier}). These equations describe the motion of viscous fluid substances; their solution provides much of the useful information for natural hazards engineering problems, e.g., flow current speed and fluid pressures on built infrastructure.

Solving the NS equations without the use of a turbulence model is often referred to as direct numerical simulation (DNS,  \cite{Orszag1970DNS}), in which the whole range of spatial and temporal scales of the turbulence must be resolved. As a result, the expense of using extremely small grid sizes and time steps is unaffordable in many practical engineering systems, especially when the Reynolds number that indicates the intensity of turbulence is high. To accommodate these issues, some variants of the NS equations are often used in practice. The two most popular approaches are Reynolds-averaged Navier-Stokes equations (RANS equations, \cite{Reynolds1895RANS, Chou1945RANS}) and large eddy simulation (LES, \cite{Deardorff1970LES}). The two models, while still able to give satisfactory approximations to the turbulence in the fluid, are much cheaper than DNS because they use new model equations to describe turbulent energy dissipation. These equations arise from the addition of new variables into the modified governing equations that result from the averaging and filtering processes to estimate turbulent phenomena. These unknown variables are typically related to known quantities such as the velocity gradient.  
While the above methods use Eulerian methods, Lagrangian-based approaches such as Smooth Particle Hydrodynamics (SPH,  \cite{Gingold1977SPH, Lucy1977SPH, Lind2020SPH}) have gained some popularity  for solving flows around complex geometries.  Other recent alternatives such as the Lattice Boltzmann Method (LBM,  \cite{Chen1998LBM}) or Immersed Boundary Method (IBM,  \cite{Peskin1972IBM, Peskin1977IBM, Peskin2002IBM}), while less widely used, can be used to take advantage of specific physical phenomena (e.g., using IBM to consider flow around complex moving boundaries).  
 
\section{Input and Output Data}
\label{sec:resp_cfd_water_io}

In all situations where CFD is used, the two basic inputs are the boundary conditions and initial conditions. The boundary conditions refer to the boundaries of the computational domain, which may include a wall where water cannot penetrate or an outlet where fluids flow out. For problems in natural hazards engineering, such boundaries can include the ground over which the fluid flows, the outside walls of a building that will affect flow path of the water, or the complex geometry of a bridge hit by a tsunami or storm surge. The initial conditions refer to the state of the fluids before the simulation starts. For instance, in tsunami modeling, the initial conditions for some nearshore regions might have all water at rest at sea level while somewhere in the ocean, a large volume of water is placed above sea level that represents a tsunami wave. Implementation of these initial conditions is not trivial, especially along the inlet boundary, and care must be taken to maintain a stable solution as time marches forward.  CFD solvers predict how the water volume and velocity evolve in time from their initial state. Different initial conditions will give different states later in time.

The output from CFD depends on the equations that are solved but generally includes water velocities, water pressure, and height of water surface (with extra treatment added to the NS equations, e.g., coupling the volume of fluid methods \citep{Brackbill1992VOF, Hirt1981VOF, Jasak1996VOF, Ubbink1997VOF, Ubbink2002VOF} at any specified moment during the simulation and at any location within the domain. The pressure field can be further processed to obtain forces on structures.

\section{Models and Software Systems}
\label{sec:resp_cfd_water_methods}

The implementation of CFD algorithms is often very complex. Many CFD software systems are developed and maintained by either commercial companies or large development teams with support from user communities. Some popular commercial software include: \citeprgm{STAR-CCM}, \citeprgm{ansysfluent}, and \citeprgm{COMSOL}. Commercial software often provide the ability to customize solvers (to some extent) by allowing user-defined functions. Many researchers prefer the complete freedom of modifying the source code and use open-source CFD software, such as \citeprgm{OpenFOAM} and \citeprgm{SU2}. By far, OpenFOAM is the most widely used and provides very comprehensive functionality in all areas of CFD. The relevant research communities have also contributed many pre-processing and post-processing tools for OpenFOAM, e.g., wave-generation tools that are often required in hazard modeling. Customized versions of OpenFOAM for certain applications are also available. Some examples include \citeprgm{HELYX}, \citeprgm{olaFlow}, and \citeprgm{IHFOAM}. The latter two are specially designed to simulate coastal, offshore, and hydraulic-engineering processes \citep{Higuera2013OlaFlow, Higuera2013bOlaFlow, Higuera2014OlaFlow, Higuera2014bOlaFlow, Higuera2015OlaFlow}.

\section{Major Research Gaps and Needs}
\label{sec:resp_cfd_water_gaps}

CFD is a broad concept that is used as a simulation approach in many industry and research fields. Although the general-purpose commercial or open-source CFD packages can provide a broad variety of requirements, cutting-edge research in a specific area often requires tackling very specialized problems. As a result, either in-house code must be developed, or heavy modification and customization must be added to CFD packages. Fortunately, the prevalence and maturity of open-source CFD software has provided a solid foundation or is, at the very least, a very helpful resource for researchers that focus on such software development.  One drawback, however, is that there is a consistent lack of complete documentation for many of these software, creating obstacles for new users.

Another challenge is the portability and scalability of CFD software, which allows the code to run on HPC facilities. The explosive growth of computing power in the past two decades has tremendously changed many areas, allowing for the performance of simulations that were impossible in the past. Unfortunately, CFD software, especially in-house code or customized solvers, may not run naturally on new machines.  OpenFOAM specifically has been shown to have difficulty scaling to maximize performance on HPC clusters. Different implementations in the code must be taken and even new algorithms designed such that the code can run efficiently on a cluster that consists of thousands of computing nodes and a cluster that consists of new architectures like GPUs.

In terms of modeling physical processes, one significant research gap is in robust modeling of the air-fluid-structure interaction phenomena associated with wave impacts and fast moving flows around complex geometries.  The effects of air compressibility and potential air entrapment are often either poorly considered or neglected altogether.  Some loosely coupled one-way approaches for fluid-structure interaction response are available, but a more comprehensive two-way coupling approach to combine CFD with various structural analysis techniques would be needed.  Similar novel coupling mechanisms between CFD solvers and the 2D approaches presented in Chapter 8 are also a notable research gap.  Because of the computational expense of CFD approaches relative to the nonlinear shallow water or Boussinesq approaches used at very large scales, approaches that allow flow to move between these types of solvers is a significant challenge.
