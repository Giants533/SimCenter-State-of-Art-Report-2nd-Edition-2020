%%%%%%%%%%%%%%%%%%%%%%preface.tex%%%%%%%%%%%%%%%%%%%%%%%%%%%%%%%%%%%%%%%%%
% sample preface
%
% Use this file as a template for your own input.
%
%%%%%%%%%%%%%%%%%%%%%%%% Springer %%%%%%%%%%%%%%%%%%%%%%%%%%

\preface

This report is a product of the NSF NHERI SimCenter and provides an overview and review of simulation requirements and software tools for natural hazards engineering of the built environment. The simulations discussed in this report are an essential component of research to address the three grand challenge areas and associated research questions outlined in the NHERI Science Plan (2017). As outlined in the NHERI Science Plan, the grand challenges entail: (1) quantifying natural hazards and their effects on civil infrastructure; (2) evaluating the vulnerability of civil infrastructure and social vulnerability of populations in at-risk communities; and (3) creation of technologies and tools to design and implement measures to promote resilience to natural hazards. Accordingly, required simulation technologies encompass a broad range of phenomena and considerations, from characterization and simulation of natural hazards and their damaging effects on buildings and civil infrastructure, to quantifying the resulting economic losses, disruption, and other consequences on society. Ultimately, the goal is to enable high-fidelity and high-resolution models in regional simulations that can support technological, economic, and policy solutions to mitigate the threat of natural hazards.

The natural hazards addressed in this report include earthquakes, tsunami, storm and tornado winds, and storm surge. While not an exhaustive list of all possible natural hazards, these are the hazards addressed under the U.S. National Science Foundation’s (NSF) NHERI research program. The report is organized in a sequential fashion, including: (1) simulation methods to characterize the natural hazards; (2) response simulation of structural and geotechnical systems and localized wind and water flows; and (3) quantifying the resulting damage and its effects on the performance of buildings, transportation systems, and utility infrastructure systems. Given the inherent uncertainties in all aspects of natural hazards engineering, methods of uncertainty propagation are reviewed, with an eye toward their broad applicability within and between the various simulation components.

Owing to the broad scope of the simulation topics, this state-of-art review is presented with the goal of educating and informing researchers, including both simulation tool developers and users, on key requirements and capabilities within each simulation topic. The report is also a guide for the development of simulation capabilities by the NSF NHERI SimCenter. Each section of the report begins with a brief overview of the purpose of the simulation component, including a discussion of the goals of the analysis (what is being calculated), the underlying physics or principles involved in the simulation, common modeling assumptions and simplifications, and typical input and output of the simulations. 

With the aim to take stock of computational simulation capabilities, inform research by the NHERI research community, and position the work of the NHERI SimCenter, the summaries identify and review commonly used simulation software that is widely known and used for research in academia and industry. Particular emphasis is placed on open-source or other software that is hosted on DesignSafe or is otherwise easily accessible to researchers, and a summary table of the simulation software tools is provided as an appendix to the report. In addition to summarizing the state-of-art in the various topic areas, each section of the report identifies major research gaps and needs, with the intent that these could motivate research proposals to NSF or other agencies that will lead to future advancements. The final chapter of the report summarizes how tools being developed by NHERI SimCenter are advancing the state-of-art in simulating the effects of natural hazards on the built environment.

This report is intended to be a living document series, which will be updated regularly based on feedback from the research community and advancements in simulation technologies for natural hazards engineering.

\vspace{\baselineskip}
\begin{flushright}\noindent
Stanford University,\hfill {\it Gregory G. Deierlein}\\
October, 2020\hfill {\it Adam Zsarnóczay}\\
\end{flushright}


