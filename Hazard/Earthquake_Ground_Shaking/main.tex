%%%%%%%%%%%%%%%%%%%% author.tex %%%%%%%%%%%%%%%%%%%%%%%%%%%%%%%%%%%
%
% sample root file for your "contribution" to a contributed volume
%
% Use this file as a template for your own input.
%
%%%%%%%%%%%%%%%% Springer %%%%%%%%%%%%%%%%%%%%%%%%%%%%%%%%%%%%%%%%%

\title{Earthquake - Ground Shaking}
% Use \titlerunning{Short Title} for an abbreviated version of
% your contribution title if the original one is too long
\author{
    \textbf{Adam Zsarnóczay}
    \and Jack Baker 
    \and Wael Elhaddad
    \and David McCallen}
\tocauthor{}
\authorrunning{Zsarnóczay et al.}
% Use \authorrunning{Short Title} for an abbreviated version of
% your contribution title if the original one is too long
%\institute{Name of First Author \at Name, Address of Institute, %\email{name@email.address}
%\and Name of Second Author \at Name, Address of Institute %\email{name@email.address}}
%
% Use the package "url.sty" to avoid
% problems with special characters
% used in your e-mail or web address
%
\maketitle

The purpose of using the software tools presented in this section is to characterize the ground shaking intensity due to earthquakes at a given site or region. Although the amount of historical earthquake data at any particular site is small, the improvements in the field since the seminal paper on earthquake hazard by \cite{cornell1968engineering} allows engineers to combine the available information from several sites and create a seismic hazard model that can quantify the expected ground-motion hazard and corresponding uncertainty. 

Probabilistic Seismic Hazard Assessment (PSHA) and disaggregation of the calculated hazard (Bazzurro and Cornell, 1998) are the most popular methods to characterize ground shaking at a site of interest. The earthquake engineering community is fortunate to have access to a large number of free--and often open-source tools--available for this task. Besides PSHA tools, also presented are tools that perform scenario-based deterministic analysis. Both probabilistic and deterministic analyses are typically based on Ground Motion Prediction Equations (GMPEs)-empirical functions that describe the attenuation of shaking intensity with increasing distance from the hypocenter. Note: the Ground Motion Prediction Model (GMPM) denomination is preferred over GMPE in several recent publications to emphasize that modern attenuation relationships are more complex than a single equation. Other approaches, such as physics-based simulation of ground motions, are also becoming sufficiently robust to be widely applicable for risk assessment and their widespread use is likely to increase in the near future. 

This section provides an overview of the above methods, the data they use, and the tools that have implemented them.

\section{Input and Output Data}
\label{sec:eq_shake_io}

\subsection{Input Data}

Ground-motion hazard assessment requires information about the site and the seismic source(s) in its vicinity. These data can be classified as follows:
% \newline

\paragraph{Site location(s)} These are latitude and longitude pairs for each site of interest. For regional analyses, the ground-motion hazard characteristics at the nodes of a carefully sized lattice grid are used, and the site-specific results are determined by interpolation.
% \newline

\paragraph{Site data} Local soil conditions have significant influence on the ground motion at the surface at a particular site. Two neighboring sites with practically identical bedrock hazard might experience fundamentally different surface ground motions if their soil characteristics are different. 

The difficulty in determining actual soil conditions is that the bedrock is often at large depths, and in the absence of site-specific characterization, there is usually limited data to characterize the complete soil profile. This is especially the case for large regional simulations. The lack of data translates into significant uncertainty in the resulting ground-motion estimate. Regional analyses typically describe soils using estimates of either the soil class (e.g., rock, stiff soil, soft clay, etc.) or the average shear-wave velocity over the top 30 meters of the soil (vs30). The USGS provides estimates of vs30 from topographic slope that can be used as an approximate solution when no other data is available \citep{usgs2020vs30}. 

\paragraph{Seismic sources} Ground motions are generated by seismic sources. Depending on the available historical in-formation about the earthquakes in the region, sources might be described as faults, points, or areas with homogeneous seismicity. The abundant information about earthquakes in California allows researchers to develop a detailed map of faults for the state \citep{field2014uniform}, while the Central and Eastern United States (CEUS) is covered by area sources \citep{mueller2015seismic}.

Scenario-based analysis typically requires a hypocenter location, earthquake magnitude, and information about the rupture surface and style of faulting. Probabilistic assessments consider all sources that might affect the region along with a stochastic description of those sources using magnitude occurrence rates, hypocenter depth distributions, etc. These stochastic models are usually based on historical ground-motion data. The Uniform California Earthquake Rupture Forecast version 3.0 (UCERF3, \cite{field2014uniform}) is an example of such a stochastic seismic-source model. It is published jointly by the United States Geological Survey (USGS), the California Geological Survey (CGS), and the Southern California Earthquake Center (SCEC). Seismic-source models for other non-US regions have been prepared and made publicly available by the Global Earthquake Model (GEM) initiative (\url{https://www.globalquakemodel.org/}). These include Europe \citep{giardini2014mapping}, the Middle East \citep{danciu20172014}, and South America \citep{garcia2018creation}.

\paragraph{Ground-motion model} The ground-motion model describes the propagation of ground shaking from the earthquake rupture surface to the collection of sites of interest. Ground-Motion Prediction Equations are the commonly used tools for this purpose. They estimate severity of ground shaking in the form of IMs. These estimates are typically based on regression to historical IM data from recorded earthquake ground motions. Depending on the data and the functional form used for the regression, one might arrive at various GMPEs. There are hundreds of GMPEs available \citep{douglas2018ground}. It is important to select the one that is based on data and assumptions matching the seismicity in the region of interest.

\paragraph{Logic tree} Logic trees have become a popular means to consider the epistemic uncertainty in the ground-shaking hazard. Branches of the trees are populated with various modeling assumptions (e.g., GMPEs, seismic-source models, maximum magnitudes, site data, etc.) and a set of weights are defined so that the branches and weights represent a probability distribution on the uncertain issue. Notwithstanding the problems inherent in such a strategy for uncertainty quantification \citep{bommer2008use}, recent research has shown several examples where the logic-tree approach provides additional insight that would otherwise not be available \citep[see e.g.,][]{goulet2017ngaeast}

\subsection{Output Data}

\noindent One or more of the following outputs are produced to describe the ground-shaking hazard:

\paragraph{Seismogram} A plot of ground motion versus time, which would be recorded by a seismometer or other instrument. Synthetic seismograms can also be generated by physics-based ground-motion simulations or stochastic models. Seismograms are typically expressed in terms of ground acceleration, which are integrated to obtain ground-motion velocity or displacement. 
% \newline

\paragraph{Intensity measure (IM)} A measure of the intensity of shaking associated with a particular seismogram. Examples of IMs are the peak ground acceleration (PGA) and spectral acceleration at a given vibration period [Sa(T)].
% \newline

\paragraph{Response spectra} A response spectrum is often used as a proxy to describe the ground-motion intensity. It is a collection of IMs that describe the response of single-degree-of-freedom oscillators to the ground-motion record of interest. Acceleration response spectra, for example, are defined through a set of Sa(T) values. The larger the number of vibration periods considered, the more detailed the resulting spectrum becomes. In GMPE-based simulations, the resolution of the response spectrum is typically limited by the vibration-period discretization used in the attenuation relationship. In physics-based simulations, the response spectrum is determined from the simulated ground-motion seismogram.
% \newline

\paragraph{Hazard curve } Rather than focusing on ground motions from a single event, probabilistic assessments characterize the hazard through the annual exceedance rate of various IMs at the site of interest. This information is represented by a hazard curve. Each IM has its corresponding hazard curve. Studies typically focus on obtaining hazard curves for a set of Sa(T) IMs. 
% \newline

\paragraph{Hazard spectrum} Given a set of hazard curves corresponding to Sa(Ti) for Ti in a sufficiently wide range of vibration periods, it is possible to collect the Sa(Ti) corresponding to a pre-defined annual exceedance rate. These Sa(Ti) values describe the so-called uniform hazard spectrum (UHS), which is the collection of spectral accelerations that have the same (uniform) probability of exceedance. An alternative to the UHS is the so-called Conditional Spectra (CS), where the spectra are conditioned on the probability of exceedance of response at a specified period T. The UHS and CS are often used to describe the hazard in a probabilistic framework.
% \newline

\paragraph{Hazard map} Given a particular annual exceedance rate (or return period, or exceedance probability over a pre-defined time period) and a corresponding hazard curve at various sites in a region, one can create a map of IM levels that describe the intensity of expected ground shaking. The engineering community uses a small set of pre-defined return periods (such as 475 years, which is equivalent to 10\% probability of exceedance over 50 years) to describe the hazard for structural design and performance assessment purposes. Using the same return periods within the community facilitates the comparison of hazard maps among different regions and within different parts of the same region.
% \newline

\paragraph{Disaggregation of the hazard} Probabilistic Seismic Hazard Analysis aggregates the contributions of several sources to produce a hazard curve for a site. Disaggregation provides a break-down of the contribution of various seismic sources to the total seismic hazard. This allows researchers to estimate the characteristic features of the earthquake scenarios (typically magnitude and source-to-site distance) that are the main contributors to the seismic hazard at the site. Information about such features complements the estimated IMs and provides a more detailed understanding of the seismic hazard at the site.

\section{Modeling Approaches}
\label{sec:eq_shake_models}

Two main approaches are typically used in the research community for ground-shaking hazard characterization:
% \newline

\paragraph{Estimate IMs using GMPEs} This type of methodology describes the hazard using IMs estimated using GMPEs that are based on historical earthquake data. The advantage of using GMPEs is the computational efficiency of the calculation; GMPEs typically describe IMs as random variables. This approach allows engineers to estimate the probability of exceeding a pre-defined IM level given the characteristics of the earthquake scenario, the location of the site, and other parameters (soil conditions, damping, etc.). By aggregating these exceedance probabilities and the occurrence rates of corresponding earthquake scenarios over a region, engineers can arrive at the total probability of exceedance of a pre-defined IM level at the site of interest. Performing such a calculation for multiple IM levels produces the hazard curve for the site. The ground shaking hazard is commonly described using a set of hazard curves that correspond to the spectral acceleration at various vibration periods. These hazard curves can serve as the basis of a UHS or CS for structural response estimation.
% \newline

\paragraph{Estimate ground-motion records using physics-based wave propagation} This type of methodology relies on a physical model of the crust and propagation of seismic waves in that model. It requires a significant amount of information about local geology to arrive at reliable results. Furthermore, the calculations are computationally expensive and usually require a High Performance Computing (HPC) environment. To the extent that the physics-based models are validated and based on reliable model parameters (geologic data), they can represent local geologic effects (such as deep geologic basins) that are not captured as well by empirical GMPEs. The earthquake simulations also provide ground-motion records directly as opposed to the IM proxies used in conventional GMPE approaches. These records can be applied directly in response estimation, which removes part of the uncertainty and ambiguity associated with GMPEs and ground-motion record selection to match IM spectra. There are computational and modeling challenges before this approach becomes commonplace, but early-adopters in the research community (such as SCEC, \url{https://www.scec.org/} , Lawrence Livermore National Laboratory, \url{https://www.llnl.gov}, and USGS, \url{https://earthquake.usgs.gov/} ) are providing physics-based simulations that are being applied for performance and risk assessment.

\section{Research Gaps and Opportunities}
\label{sec:eq_shake_gaps}

As our ability to compute ground motions to ever higher frequencies increases as a result of rapid computational advancements, we are outstripping our knowledge of the subsurface geology at fine scale. A challenge will be how to address geologic model uncertainties. [credit: Dave McCallen]

\section{Software and Systems}
\label{sec:eq_shake_tools}

The following is a list of software that is commonly used for characterizing earthquake ground shaking.

\paragraph{AWP-ODC} \citeprgm{AWP-ODC} is an elastic wave propagation program (AWP-ODC, \cite{cui2010scalable}) that performs a parallel finite-difference wave-propagation simulation. The software can simulate the dynamic rupture and wave propagation that occurs during an earthquake. It was originally written in Fortran and supports parallel computation using the message passing interface (MPI). A version of the software written in C and CUDA is also available to run on graphics processing units (GPUs).

\paragraph{BBP} \citeprgm{The Broadband platform} \citep{maechling2015scec} is a software system developed by SCEC that can be used to generate synthetic ground motions using wave-propagation models. The BBP is distributed with data products (velocity models and Green's functions packages) that allows for the generation of seismograms for simulating historical or hypothetical earthquake scenarios in California, the northeast of the U.S., and Japan. The software runs on Linux systems and provides different seismogram generation models.

\paragraph{CyberShake} \citeprgm{CyberShake} is a computational simulation project on physics-based PSHA, developed and hosted by SCEC. CyberShake simulations have been created from studies that define the in-puts, computational software, and the outputs. Outputs from studies done using CyberShake are stored in publicly accessible databases, which includes studies performed for southern and central California. Data products of CyberShake include seismograms, hazard curves, dis-aggregation, duration results, and hazard maps.

\paragraph{HAZ} \citeprgm{HAZ} is a probabilistic seismic hazard analysis tool written and maintained by Norman Abrahamson. It is written in FORTRAN and it is not heavily documented, but it still is a popular tool for several bridge, dam, and nuclear projects.
%latest activity on the HAZ repository was in 2017.

\paragraph{HAZUS 4.2} The Federal Emergency Management Agency (FEMA) has been supporting the development of this tool for more than two decades. It is publicly available and provides a convenient way to perform regional risk assessment following the HAZUS Multi-hazard Loss Estimation Methodology 2.1 \citep{fema2011hurricanetechnical,fema2011earthquaketechnical,fema2011floodtechnical,fema2017tsunamitechnical}. The methodology covers earthquake, hurricane, tsunami, and flood hazards %(with tsunami currently under development)
. Researchers and agencies in the U.S. can download input data with the tool that provides in-formation about the hazard, the exposure (i.e., building locations and characteristics) and the vulnerability (i.e., building fragility and consequence functions) of the region. These inputs are prepared in the standard format required by the software and provide exposure data (building inventories, etc.) at a census-tract-level resolution. The software runs on Microsoft's Windows operating system and has a GUI.
% \newline

\paragraph{Hercules} Hercules \citep{tu2006mesh} is a parallel finite-element wave-propagation software that can be used to simulate earthquake ruptures. It was originally developed by the Quake group in Carnegie Mellon University in collaboration with SCEC. The software is designed to be memory-efficient and highly scalable to run large-scale simulations in an HPC environment \citep{taborda2010speeding}. 
% \newline

\paragraph{NSHMP-Haz} NSHMP-Haz is a Java-based library for PSHA that has been developed as part of the National Seismic Hazard Mapping Project (NSHMP) within the USGS Earthquake Hazards Program (EHP). The library is the engine driving different USGS web services and applications, and it enables high-performance seismic hazard calculations required for generating hazard maps over large regions using different ground-motion models. A legacy Fortran version of this library is also available, although it is deprecated at the time of writing this report.
% \newline

\paragraph{OpenSHA} OpenSHA \citep{field2003opensha} is an open-source platform for seismic hazard analysis (SHA) that was developed by the SCEC in collaboration with USGS. The platform is comprised of Java libraries and a suite of applications that are suitable for different SHA applications. For instance, OpenSHA provides graphical applications for calculating hazard spectra, hazard curves, hazard maps, hazard disaggregation, and querying site data. In addition, all the features provided by the graphical applications are available programmatically through the OpenSHA Java libraries.
% \newline

\paragraph{OpenQuake Engine} OpenQuake \citep{pagani2014openquake} is an open-source library for seismic hazard and seismic risk computations. The library was developed using Python and is cross platform. The library uses data, methods, and guidelines outlined by GEM.
% \newline

\paragraph{PEER Ground Motion Database} The PEER ground motion database (NGA-West2, \cite{ancheta2014ngawest2}) is a comprehensive set of ground-motion records from shallow crustal earthquakes in active seismic regions around the world. The database includes 21,336 records from 599 earthquake events. In addition to the ground-motion records, the database stores detailed metadata that includes different source-site distance measures, site, and rupture characterization. A web service that can perform ground-motion record selection and scaling using the records in database is also provided.

\paragraph{R-CRISIS } \citeprgm{R-CRISIS} is the latest version of the CRISIS software that has been developed by \cite{ordaz2013crisis2008} and supported by the National Autonomous University of Mexico, and the Italian Department of Civil Protection. The latest version of the software is free and publicly available. It is designed to work with a GUI in a Windows environment. The software is designed to per-form PSHA calculations with a large number of GMPEs built in and seismic-source models available in the literature.
% \newline

\paragraph{SW4} Seismic waves, 4th order (SW4, \cite{petersson2017geodynamics}) is a software originally developed at the Lawrence Livermore National Lab (LLNL) and currently jointly developed by the LLNL and the Lawrence Berkeley National Lab under the U.S. DOE Exascale Computing Program EQSIM application framework. It can solve three-dimensional (3D) seismic wave-propagation problems. The software was developed using Fortran and C++ and makes use of a distributed memory-programming model using MPI. The software is suitable for running on HPC and is capable of producing synthetic seismograms in different formats.
% \newline

\paragraph{UCVM} The Unified Community Velocity Model (UCVM, \cite{small2017scec}) is a software framework developed by SCEC that provide a common interface to query different 3D seismic velocity models for the State of California. The software allows the use of alternative models to query and visualize seismic-wave velocities. The software provides query scripts to obtain the seismic-wave velocities and density visualized on a horizontal slice, cross section, and depth pro-file, and can also provide basin depth and vs30 maps. The properties provided by the velocity models included with UCVM are crucial for many wave-propagation software that is present-ed in this section.
% \newline

\paragraph{UGMS MCER} UGMS MCER \citep{crouse2018sitespecific} is a web-based tool developed by SCEC Committee for Utilization of Ground Motion Simulations (UGMS) to provide a site-specific Maximum Considered Earthquake (MCE) response spectra for the Los Angeles region according to the site-specific seismic hazard analysis procedure outlined in ASCE 7-16. The tool is user-friendly and is oriented towards practitioners, and only requires the location (latitude and longitude) and site-soil classification (site class or vs30).

