%%%%%%%%%%%%%%%%%%%% author.tex %%%%%%%%%%%%%%%%%%%%%%%%%%%%%%%%%%%
%
% sample root file for your "contribution" to a contributed volume
%
% Use this file as a template for your own input.
%
%%%%%%%%%%%%%%%% Springer %%%%%%%%%%%%%%%%%%%%%%%%%%%%%%%%%%%%%%%%%


%% RECOMMENDED %%%%%%%%%%%%%%%%%%%%%%%%%%%%%%%%%%%%%%%%%%%%%%%%%%%
%\documentclass[graybox]{svmult}
%
%% choose options for [] as required from the list
%% in the Reference Guide
%
%\usepackage{mathptmx}       % selects Times Roman as basic font
%\usepackage{helvet}         % selects Helvetica as sans-serif font
%\usepackage{courier}        % selects Courier as typewriter font
%\usepackage{type1cm}        % activate if the above 3 fonts are
                             % not available on your system
%
%\usepackage{makeidx}         % allows index generation
%\usepackage{graphicx}        % standard LaTeX graphics tool
%                             % when including figure files
%\usepackage{multicol}        % used for the two-column index
%\usepackage[bottom]{footmisc}% places footnotes at page bottom
%
%% see the list of further useful packages
%% in the Reference Guide
%
%\makeindex             % used for the subject index
%                       % please use the style svind.ist with
%                       % your makeindex program
%
%%%%%%%%%%%%%%%%%%%%%%%%%%%%%%%%%%%%%%%%%%%%%%%%%%%%%%%%%%%%%%%%%%%%%%%%%%%%%%%%%%%%%%%%%%
%
%\begin{document}

\title{Electrical Transmission Substations and Lines}
% Use \titlerunning{Short Title} for an abbreviated version of
% your contribution title if the original one is too long
\author{
    \textbf{Iris Tien}
    \and {Craig Davis}
    \and {Thomas O'Rourke}
    \and {Adam Zsarnóczay}}
\tocauthor{}
\authorrunning{Tien et al.}
% Use \authorrunning{Short Title} for an abbreviated version of
% your contribution title if the original one is too long
%\institute{Name of First Author \at Name, Address of Institute, %\email{name@email.address}
%\and Name of Second Author \at Name, Address of Institute %\email{name@email.address}}
%
% Use the package "url.sty" to avoid
% problems with special characters
% used in your e-mail or web address
%
\maketitle

The objective of the simulation of electrical substations and transmission and distribution lines is similar to that of other lifelines. Namely, the objective is to assess the ability to provide critical electricity services for populations under varying scenarios. As relevant for natural hazards engineering, the purpose is to assess the state of these lifeline systems under hazard events and inform decision making on approaches to improve the expected performance of these systems under hazards. Compared to hydraulic or pressure flow analyses for water, sewer, and gas pipeline simulation, the underlying physics of the simulation relies on power voltage flow analysis in addition to system-level analyses that can be conducted through the use of network graphs, for example. Electrical power network are also composed of connected regional and local systems. Local distribution systems are dependent upon the regional grid for supply, requiring multiple levels of analysis to capture the performance of the regional grid and local distribution systems.
 
\section{Input and Output Data}
\label{sec:perf_power_io}

Modeling electrical networks requires information about generation facilities, substations, and transmission and distribution circuits. In addition to the geographical location, the level of voltage is an important property. The availability of electrical power system data is subject to many of the constraints of lifeline data, with privacy and national security concerns and the fact that the databases are privately owned. When there is a lack of network data availability, particularly a lack of detailed information at the distribution level, only low resolution analyses at the transmission system level are possible. Depending on the type of hazard, other details can be required, such as anchorage of components for seismic analysis or elevation information for flood hazard assessment. The required inputs and produced outputs for the power grid depend in part on the type of hazard, e.g., climatic compared to geologic, and the target fidelity for reliability and risk analysis. 

\section{Modeling Approaches}
\label{sec:perf_power_methods}

In general, given the IM at the site, component-specific fragility curves can be used to estimate the damage to a given component under a hazard. For facilities and building-like structures the damage evaluation is similar to the HAZUS method described in Section 3.1. Given the damage to the network component, the direct consequences are evaluated using empirical relationships based on past experience. Direct consequences are typically limited to restoration time and replacement cost in HAZUS. In HAZUS, fragility of electrical substations and distribution circuits is defined with respect to the percentage of subcomponents being damaged.

Modeling indirect consequences of damage in the electric power network is an ongoing area of research (Moore et al., 2005). In HAZUS, interaction between nodes of the power network is not considered. Interaction between other lifeline systems is considered with the following approach: the loss of electric power is assumed to influence the slight/minor and moderate DSs of components in other lifelines that depend on power. More severe DSs are not influenced by the lack of power. The substation that serves connected components is assumed to experience the event at the location of the served component. An even more simplified approach uses a generic damage algorithm to describe the availability of power as a function of an IM. 

\section{Software Systems}
\label{sec:perf_power_tools}

\noindent\textbf{HAZUS 4.2} \\As mentioned in Section 3.3, HAZUS groups electric power and communication networks with the other lifelines and provides similar methodologies for their analysis. Electrical networks are the only lifeline type that has influence on other lifelines in the HAZUS methodology (i.e., damage to the electric network and the consequent loss of power results in loss of function and potential damage in other lifelines). HAZUS 4.2 software allows estimation of lifeline damage and consequent reduction in performance. Further details of the software and its limitations are explained in Section 1.1, and its application to lifeline simulation and pipelines in particular are described in Section 3.3.
\newline

\noindent\textbf{OpenDSS} \\OpenDSS is a software package developed by the Electric Power Research Institute (EPRI). It conducts electrical power system simulation for electric utility power distribution systems. It supports simulations of power flow across frequencies and is mainly used to evaluate distributed energy resource generation, its integration with utility distribution systems, and grid modernization technologies. Assessments do not include the impacts of natural hazards on the system components or network performance for natural hazards engineering applications.
\newline

\noindent\textbf{MatPower} \\MatPower is a free toolbox available in MATLAB for conducting power flow analysis. It does not, however, have the capability to directly model natural hazard effects.

\section{Research Gaps and Needs}
\label{sec:perf_power_gaps}

Similar to the description of the research gaps and needs in water, sewer, and gas pipeline simulation (Section 3.3.4), there is the need for improved consideration of interdependencies between lifelines as related to electrical systems. Given the criticality of electricity for many lifelines, this will enable researchers to better understand the impacts of natural hazards on communities. Advancing component-level fragility curves and improved simulation of structural response for critical facilities will lead to a better estimate of expected damage, both in terms of accuracy and resolution in a simulation. Needed are improved fragility functions for specialized structures for each lifeline sector, e.g., generation plants, substations and components, and transformers for electrical systems. Analysis of equipment supporting these specialized facilities is also needed. The focus on selected structural components in electrical power network damage analysis, where fragility functions do exist, can also lead to missing failure mechanisms and inaccurate risk assessment. Finally, improved simulation of the recovery process will enable researchers to better understand performance and recovery of lifeline services during and after disasters. The high computational cost and sometimes computational intractability of network-level performance analyses for electrical systems is an ongoing research challenge.

\section{References}

% TODO: These are the references that need to be added to the references.bib file

% Applied Technology Council (1985) ATC-13: Earthquake Damage Evaluation Data for California, ATC, Redwood City, CA, USA

% ARUP (2013) Resilience-based Earthquake Design Initiative for the Next Generation of Buildings, ARUP, USA

% Attary N., Unnikrishnan V.U., van de Lindt J.W., Cox D.T., Barbosa A.R. (2017) Performance-Based Tsunami Engineering methodology for risk assessment of structures, Engineering Structures, 141:676-686

% Baker JW, Lin T, Shahi SK, Jayaram N (2011). New ground motion selection procedures and selected motions for the PEER transportation research program. PEER Report 2011/03, Pacific Earthquake Engineering Research Center, University of California, Berkeley, CA.

% Barbato M., Petrini F., Unnikrishnan V.U., Ciampoli M. (2013) Performance-Based Hurricane Engineering (PBHE) framework, Structural Safety 45: 24-35, doi: 10.1016/j.strusafe.2013.07.002

% Bradley B.A., Lee D.S., Broughton R., Price C. (2010) Efficient Evaluation of Performance-Based Earthquake Engineering Equations

% Bradley B.A. (2013) A critical examination of seismic response uncertainty analysis in earthquake engineering, Earthquake Engineering and Structural Dynamics, 42:1717-1729

% Carey, T.J., Mason, H.B., Barbosa, A.R., and Scott, M.H. "Multi-hazard earthquake and tsunami effects on soil-foundation-bridge systems." Journal of Bridge Engineering, Accepted, August 2018

% Caltrans—California Department of Transportation (2013). “Caltrans Seismic Design Criteria, Version 1.7.” California Department of Transportation, Sacramento, CA.

% Chang SE, Shinozuka M, Moore JE. Probabilistic Earthquake Scenarios: Extending Risk Analysis Methodologies to Spatially Distributed Systems. Earthq Spectra 2000;16:557–72.

% Chmielewski H., Guidotti R., McAllister T., Gardoni P. (2016) Response of Water Systems under Extreme Events: A Comprehensive Approach to Modeling Water System Resilience, In: Proc. 16th World Environmental and Water Resources Congress, 475-486

% Choi E., DesRoches R., Nielson B. (2004) Seismic fragility of typical bridges in moderate seismic zones, Engineering Structures, 26:187-199

% Dueñas‐Osorio, L., Craig, J. I., & Goodno, B. J. (2007). Seismic response of critical interdependent networks. Earthquake engineering & structural dynamics, 36(2), 285-306.

% G&E Engineering Systems, Inc. (G&E), NIBS Earthquake Loss Estimation Methods, Technical Manual, (Water Systems), May 1994.

% Fajfar P., Krawinkler H. (ed.) (2004) Performance-Based Seismic Design Concepts and Implementation, Proceedings of an International Workshop, Bled Slovenia

% Fan, C., Zhang, C., Yahja, A., & Mostafavi, A. (2019). Disaster City Digital Twin: A vision for integrating artificial and human intelligence for disaster management. International Journal of Information Management, 102049.

% FEMA Mitigation Division (2017) HAZUS – Tsunami Model Technical Guidance, FEMA, Washington D.C., 183p

% FEMA Mitigation Division (2018a) Hazus 4.2 software, FEMA, Washington D.C., 2018

% FEMA Mitigation Division (2018b) HAZUS – Multi-hazard Loss Estimation Methodology 2.1, Earthquake Model Technical Manual, FEMA, Washington D.C., 718p, (Accessed: 14 Nov. 2018)

% FEMA Mitigation Division (2018c) HAZUS – Multi-hazard Loss Estimation Methodology 2.1, Hurricane Model Technical Manual, FEMA, Washington D.C., 1456p, (Accessed: 14 Nov. 2018)

% FEMA Mitigation Division (2018d) HAZUS – Multi-hazard Loss Estimation Methodology, Flood Model Technical Manual, FEMA, Washington D.C., 569p, (Accessed: 14 Nov. 2018)

% FEMA (2012) Seismic Performance Assessment of Buildings Volume 1 – Methodology, FEMA P58-1, FEMA, Washington D.C.

% FEMA (2018) Seismic Performance Assessment of Buildings Volume 3 – Performance Assessment Calculation Tool (PACT) Version 2.9.65, FEMA, Washington D.C.

% Han Y, Davidson RA. Probabilistic seismic hazard analysis for spatially distributed infrastructure. Earthq Eng Struct Dyn 2012;41:2141–2158. doi:10.1002/eqe.2179.

% He, X. (2019). Disaster risk management of interdependent infrastructure systems for community resilience planning (Doctoral dissertation, University of Illinois at Urbana-Champaign).

% Isoyama R. and Katayama T. (1982) Reliability Evaluation of Water Supply Systems During Earthquakes

% Jacob, K., Deodatis, G., Atlas, J., Whitcomb, M., Lopeman, M., Markogiannaki, O., Kennett, Z., Morla, A., Leichenko, R. and Vancura, P. (2011). “Responding to Climate Change in New York State: The ClimAID Integrated Assessment for Effective Climate Change Adaptation in New York State: Transportation,” Annals of the New York Academy of Sciences, Vol. 1244, No. 1, pp. 299-362.

% Johansen C., Horney J., and Tien I. (2016) Metrics for evaluating and improving community resilience, 23(2). ASCE Journal of Infrastructure Systems. doi:10.1061/(ASCE) IS.1943-555X.0000329

% Johansen, C., & Tien, I. (2018). Probabilistic multi-scale modeling of interdependencies between critical infrastructure systems for resilience. Sustainable and Resilient Infrastructure, 3(1), 1-15.

% Johnson, L. and T.D. O’Rourke (2016) “Critical Assessment of Lifeline System Performance: Understanding Societal Needs in Disaster Recovery”, NIST CGR 16-917-39, Applied Technology Council, Redwood City, CA. 

% Jun, L. and Y. Guoping (2013) “Iterative Methodology of Pressure-Dependent Demand Based on EPANET for Pressure-Deficient Water Distribution Analysis” Journal of Water Resources Panning and Management, ASCE, 139(1), 34-44. 

% Karaman H, Şahin M, Elnashai AS (2008). Earthquake loss assessment features of Maeviz-Istanbul (Hazturk). Journal of Earthquake Engineering, 12:175-186

% Kaviani P, Zareian F, Taciroglu E (2012). “Seismic behavior of reinforced concrete bridges with skew-angled abutments,” Engineering Structures, 45, 137-150.

% Kiremidjian AS, Moore J, Fan YY, Basoz N, Yazali O, Williams M (2006) Pacific Earthquake Engineering Research Center Highway Demonstration Project. PEER 2006/02

% Klise K.A., Moriarty D., Bynum M.L., Murray R., Burkhardt J., and Haxton T.M., Water Network Tool for Resilience (WNTR) User Manual, U.S. Environmental Protection Agency, Washington D.C., EPA/600/R-17/264, 2017

% Lange D., Devaney S., Usmani A. (2014) An application of the PEER performance based earthquake engineering framework to structures in fire, Engineering Structures, 66:100-115

% Lee R, Kiremidjian AS (2006) Uncertainty and Correlation for Loss Assessment of Spatially Distributed Systems, under review Earthquake Spectra

% Loth C., and Baker J.W. (2013) A spatial cross-correlation model of ground motion spectral accelerations at multiple periods. Earthquake Engineering & Structural Dynamics, 42(3), 397-417

% Mangalathu S, Jeon JS, Padgett JE, DesRoches R (2016). ANCOVA-based grouping of bridge classes for seismic fragility assessment, Engineering Structures, 123, 379-394.

% Mangalathu S, Soleimani F, Jeon J-S (2017). Bridge classes for regional seismic risk assessment: improving HAZUS models, Engineering Structures, 148, 755-766.

% Markov I., Grigoriu M., and O'Rourke T., An evaluation of Seismic Serviceability of Water Supply Networks with Application to San Francisco Auxiliary Water Supply System, NCEER Report No. 94-0001, 1994.

% McKenna F, Zsarnóczay A., Elhaddad W., Performance Based Engineering Application User Manual, 2018

% Moore JE, Little RG, Cho S, Lee S (2005) Using Regional Economic Models to Estimate the Costs of Infrastructure Failures: The Cost of a Limited Interruption in Electric Power in the Los Angeles Region. Keston Institute for Infrastructure, University of Southern California

% New York City Department of Environmental Protection (2013), “New York City Wastewater Resiliency Plan” https://www1.nyc.gov/assets/dep/downloads/pdf/climate-resiliency/climate-plan-single-page.pdf 

% O’Rourke, T.D. (2010) “Geohazards and Large Geographically Distributed Systems”, 2009 Rankine Lecture, Geotechnique, Vol. LX, No. 7, July, 2010, pp. 503- 543.

% O’Rourke, T.D., (2014) “Earthquake-Resilient Lifelines: NEHRP Research, Development and Implementation Roadmap” NIST GCR 14-917-33, NEHRP Consultants Joint Venture, Redwood City, CA. 

% Ouyang M. Review on modeling and simulation of interdependent critical infrastructure systems. Reliability Engineering and System Safety, 121, 43–60. doi:10.1016/j. ress.2014.06.040, 2014

% M. Pagani, D. Monelli, G. Weatherill, L. Danciu, H. Crowley, V. Silva, P. Henshaw, L. Butler, M. Nastasi, L. Panzeri, M. Simionato and D. Vigano, OpenQuake Engine: An Open Hazard (and Risk) Software for the Global Earthquake Model, Seismological Research Letters, vol. 85, no. 3, pp. 692-702, 2014.

% Park Y-J, and Ang A. H-S. (1985) Mechanistic seismic damage model for reinforced concrete. Journal of structural engineering 111.4: 722-739.

% Pitilakis K, Franchin P, Khazai B, Wenzel H (eds) (2014). SYNER-G: Systemic seismic vulnerability and risk assessment of complex urban, utility, lifeline systems and critical facilities. Methodology and applications. Geotechnical, Geological and Earthquake Engineering, 31, ISBN 978-94-017-8834-2, Springer, The Netherlands.

% Porter K.A., Kiremidjian A.S., LeGrue J.S. (2001) Assembly-Based Vulnerability of Buildings and Its Use in Performance Evaluation, Earthquake Spectra, 17:291-312

% Rathje, E., Dawson, C. Padgett, J.E., Pinelli, J.-P., Stanzione, D., Adair, A., Arduino, P., Brandenberg, S.J., Cockerill, T., Dey, C., Esteva, M., Haan, Jr., F.L., Hanlon, M., Kareem, A., Lowes, L., Mock, S., and Mosqueda, G. 2017. “DesignSafe: A New Cyberinfrastructure for Natural Hazards Engineering,” ASCE Natural Hazards Review, doi:10.1061/(ASCE)NH.1527-6996.0000246.

% Reese S., Bradley B.A., Bind J., Smart G., Power W., Sturman J. (2011) Empirical Building fragilities from observed damage in the 2009 South Pacific tsunami, Earth-Science Reviews, 107:156-173

% Romero N., O’Rourke T.D., Nozick L.K., Davis C.A. (2010) Seismic Hazards and Water Supply Performance, Journal of Earthquake Engineering, 14:1022-1043

% Rossman L.A., EPANET 2 Users Manual, U.S. Environmental Protection Agency, Cincinnati, OH, 2000

% Sayyed, M.A.H., R. Gupta, and T.T. Tanyimboh (2014) “Modelling Pressure Deficient Water Distribution Networks in EPANET”, 16th Conf. on Water Distr. Sys, Analysis, Procedia Engineering 89, 626-631. 

% Stergiou E and Kiremidjian AS (2006) Treatment of Uncertainties in Seismic Risk Analysis of Transportation Systems. The John A. Blume Earthquake Engineering Center, Report No. 156, Stanford University, Stanford, CA

% Tabucchi, T., Davidson, R., & Brink, S. (2010). Simulation of post-earthquake water supply system restoration. Civil Engineering and Environmental Systems, 27(4), 263-279.

% Tang A. and Wong F., Observation on Telecommunications Lifeline Performance in the Northridge Earthquake of January 17, 1994, Magnitude 6.6, 1994.

% Tomar, A., Burton, H. V., Mosleh, A., & Yun Lee, J. (2020). Hindcasting the Functional Loss and Restoration of the Napa Water System Following the 2014 Earthquake Using Discrete-Event Simulation. Journal of Infrastructure Systems, 26(4), 04020035.

% Wang Y., O’Rourke T.D. (2008) Seismic Performance Evaluation of Water Supply Systems, MCEER Technical Report 08-0015, MCEER, Buffalo, NY

% Werner S.D., Cho S., Taylor C.E., Lavoie J-P., Huyck C., Eguchi R.T. (2006), REDARS 2 Demonstration Project for Seismic Risk Analysis of Highway Systems, California Department of Transportation, Sacramento, California

% Zhu M, Elkhetali I, Scott MH (2018), Validation of OpenSees for Tsunami Loading on Bridge Superstructures, Journal of Bridge Engineering, 23:4

% Zsarnóczay A. (2018) PELICUN – Probabilistic Estimation of Losses, Injuries and Community resilience Under Natural disasters, https://github.com/NHERI-SimCenter/pelicun, doi: 10.5281/zenodo.1489230